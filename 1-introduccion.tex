
\chapter{Introducción}
\label{chap:intro}\label{chap:1}


\section{Qué es question answering}
\label{sec:que-es-qa}

Question answering es un área de ciencias de la computación que busca generar respuestas concretas a preguntas expresadas en algún lenguaje natural. Es un área compleja que combina herramientas de búsqueda y recuperación de la información (\textit{information retrieval}), de procesamiento del lenguaje natural (\textit{PLN}) y de extracción de información (\textit{information extraction}). Por poner un ejemplo: para el input \textit{\dq{?`Cuándo nació Noam Chomsky?}} un sistema de question answering podría devolver \dq{\textit{7 de diciembre de 1928}}.

Podríamos pensar esta área como un paso lógico posterior, o un refinamiento, de los sistemas de recuperación de información usuales (por ejemplo, los buscadores web). En estos sistemas, el input no es una pregunta en lenguaje natural sino una serie de palabras preseleccionadas por el usuario para el motor de búsquedas, y el output no es una respuesta concreta sino una serie de documentos relevantes según el sistema, que luego el usuario deberá evaluar y revisar por su cuenta para encontrar la información que quiere.

Question answering logró en los últimos años una serie de hitos impulsados por el proyecto general de la web semántica. Watson, el sistema desarrollado por IBM que derrotó a los mejores competidores de Jeropardy! en tiempo real\footnote{En esta url está disponible el programa en el que el sistema vence a sus competidores humanos: \url{http://www.youtube.com/watch?v=WFR3lOm_xhE}} es el ejemplo más visible, pero incluso buscadores como Bing! y Google comienzan a incorporar este tipo de algoritmia.

Los sistemas de question answering suelen ser sistemas complejos que abordan distintas problemáticas: por un lado deben definir y optimizar la base de conocimiento para el dominio dado, por otro deben realizar un análisis de las preguntas en lenguaje natural a fin de volverlas tratables computacionalmente y, finalmente, deben poder buscar -o generar- y decidir la mejor respuesta para la pregunta ingresada, si es que esa respuesta existe. Ejemplos de sistemas de question answering abundan: por un lado, los ya mencionados gigantes comerciales Google\footnote{\url{http://www.google.com}} y Bing!\footnote{\url{http://www.bing.com}} incorporan día a día mayor soporte para question answering, también Siri\footnote{\url{http://www.apple.com/ios/siri/}} y Google Now\footnote{\url{https://www.google.com/search/about/learn-more/now/}}, los asistentes personales para iPhone y Android respectivamente incorporan un motor de question answering y, también, el ya mencionado IBM-Watson\footnote{\url{http://www.ibm.com/watson/}}. Por otro lado, existen desarrollos no comerciales ni tan conocidos de envergadura como, por ejemplo, el proyecto de la universidad Carnegie Mellon,  Ephyra\footnote{\url{http://www.ephyra.info/}}\cite{EPHYRA1} (ahora discontinuado), el sistema basado en web START\footnote{\url{http://start.csail.mit.edu/}}, del MIT, WolframAlpha\footnote{\url{www.wolframalpha.com/}}, LASSO y Falcon \cite{QA1}\cite{QA3}, YagoQA \cite{YAGO-QA1}, Qanus \cite{QANUS1} entre una infinidad de otros trabajos destacables.

Algunos de estos subproblemas tienen nombre propio en la literatura y son una sub-área específica. Por ejemplo, dependiendo de la amplitud de la base de conocimiento, un sistema es de dominio cerrado (\textit{closed domain}), si la base es acotada a un dominio de la realidad específico; por el contrario se llama de dominio abierto (\textit{open domain}), si no lo es, es decir, si se espera que sepa responder preguntas de cualquier dominio. Por su parte, dominios de conocimiento más pequeños, en general, requieren y permiten un modelado más exhaustivo
de los datos y un análisis más estructurado, mientras que dominios de conocimiento más abiertos suelen
tener un enfoque apoyado más fuertemente en el análisis lingüístico cuantitativo.

Otra distinción usual contempla el tipo de datos de la base de conocimiento:  puede ser estructurado, como en una base de datos
relacional, semi-estructurado, como los documentos XML, o también sin estructura, como el texto plano. Cada tipo de datos tiene su enfoque:
los datos estructurados definen una ontología acotada que limita qué cosas se pueden preguntar y, en consecuencia, qué cosas se pueden responder: el problema en este caso consiste en traducir la pregunta formulada en un lenguaje humano a una consulta válida definida en el modelo de datos. Por otro lado, si los datos no tienen estructura, no es posible definir una ontología rígida y se hace necesario otro tipo de enfoque más difuso y basado en análisis lingüísticos del corpus de datos mismo contra la pregunta. No siempre,  pero en general una base de conocimiento closed domain está asociada a un tipo de datos estructurado o semi-estructurado, mientras que las bases open domain suelen ser no estructuradas.

Otro tipo de clasificación se centra en el tipo de preguntas que los sistemas saben responder. Los tipos más conocidos son las factoids -o fácticas-, que refieren a hechos fácticos (¿Cuándo ocurrió ...? ¿Cuántos hijos tiene...? ¿En dónde vivía Y en el año ...?), las listas (¿Qué libros escribió Nietzsche entre 1890 y 1900?) y las definiciones (¿Qué es un huracán?). Otros tipos usuales son las preguntas por un modo (¿Cómo...?), por una razón (¿Por qué...?) y, en general, pueden agregarse subclasificaciones como dependencias temporales explícitas o dependencias semánticas con otras preguntas.

En \allref{chap:estado-de-arte} veremos con detalle los enfoques arquitecturales usuales del área y los componentes típicos de cada módulo.

\section{Proyecto de la tesis}
\label{sec:proyecto}

En esta tesis investigamos los distintos problemas que se subsumen bajo el concepto de question answering y reseñamos diferentes soluciones y modelos aplicados para resolverlos, bajo el proyecto de la implementación de dos sistemas básicos de question answering: uno de dominio cerrado, específico, y datos estructurados en inglés, por un lado, y otro sistema multilingüe, de dominio abierto y que utiliza como corpora las wikipedias de diferentes idiomas, por el otro. Para el primer modelo orientamos nuestro desarrollo de acuerdo al modelo teórico del paper \cite{QADB1} e implementamos soluciones para un conjunto restingido de preguntas. Para el segundo modelo utilizamos un subconjunto de los problemas de la competencia CLEF '07 y desarrollamos el sistema utilizando como baseline el framework Qanus\footnote{ \url{http://www.qanus.com/}}, adaptándolo para utilizar herramientas de procesamiento de lenguajes multilingües de la librería Freeling\footnote{\url{http://nlp.lsi.upc.edu/freeling/}}.

Mientras el enfoque estructurado es complicado de evaluar debido a su dominio acotado y específico, es más sencillo realizar experimentos sobre las herramientas al utilizarlas en el modelo no estructurado, gracias a las distintas competencias cuya razón de ser es, justamente, la creación de criterios de evaluación al nivel de la comunidad de investigadores del área. El aporte concreto de esta tesis son dos modelos de software con diferentes soportes para idiomas: un modelo básico de question answering como interfaz a la base de datos en inglés, montado sobre una base de datos con información de países y ciudades de ejemplo, y un modelo no estructurado de dominio abierto, completamente multilingüe, que resuelve algunas subtareas de la competencia Clef '07.


Con un poco más de detalle, la implementación del modelo de dominio cerrado está basada en los trabajos de Ana María Popescu et. al.  \cite{QADB1} y \\
\cite{QADB2} en donde se define la noción de pregunta \textit{semánticamente tratable} en el contexto de una base de datos concreta. Proponen allí, así mismo, un modelo teórico para identificar este tipo de preguntas y una forma de transformarlas en consultas de SQL, así como también un sistema concreto que implementa el modelo. Popescu argumenta que una interfaz a una base de datos en lenguaje natural puede no identificar una pregunta, pero que jamás debe mal interpretarla activamente, es decir, interpretar algo distinto a lo preguntado y dar una respuesta que no se condice con la necesidad de información del usuario. Una mala interpretación de una pregunta reducirá la confianza del usuario en el sistema, volviéndolo inusable. En cambio, identificando una pregunta como intratable, se puede disparar un proceso de reformulación o especificación asistida de la pregunta, lo cual no es tan costoso en términos de confianza en el sistema. La idea principal que guía a sus trabajos es proponer una clase de preguntas específica que sea 1) suficientemente sencilla para ser interpretada por un modelo computacional y, a la vez, suficientemente abarcadora de las preguntas generalmente hechas por humanos. La intuición detrás de este enfoque es que, mientras, en el caso general, una pregunta puede ser compleja, ambigüa y difícil de comprender (incluso por un humano), también hay preguntas simples, unívocas y con una interpretación sencilla incluso para una máquina (por ejemplo: ``¿Qué restaurantes de comida china hay en Belgrano?''). La \textit{tratabilidad semántica}, cualidad de una pregunta para una base de datos dada, define esta clase de preguntas simples. Nosotros implementamos una versión limitada de este modelo teórico en \allref{chap:4} sobre un la base de datos World\footnote{Ver \url{http://dev.mysql.com/doc/world-setup/en/index.html} y \url{http://dev.mysql.com/doc/index-other.html}} (ver \allref{sec:popescu-db}), provista por mysql, que consiste en tres tablas (countries, cities y languages), y posee cierta información básica de geografía internacional.

Por su parte, para desarrollar y evaluar mecanismos de dominio abierto resolvimos algunos ejercicios de la competencia de question answering organizada por CLEF
en 2007. CLEF (de \textit{Cross-Language Evaluation Forum}) es una organización que busca fomentar la investigación en sistemas de information retrieval cross-language. En particular, una vez por año CLEF lanza una competencia de Question Answering multilingüe, con diferentes corpora y diferentes tipos de ejercicios. Estas competencia permiten obtener un patrón estándar de comparación entre distintos desarrollos y una idea general del estado de arte alcanzado en cada área.
Por ejemplo, la competencia ya finalizada del año 2013, QA4MRE@CLEF2013, (Question Answering for Machine Reading Evaluation) se enfoca principalmente en Machine Reading, tarea que incluye un grado de razonamiento elevado para la computadora\footnote{\url{http://celct.fbk.eu/QA4MRE/}}. Existen distintas conferencias de evaluación de sistemas QA o de subtareas asociadas (por ejemplo TREC - Text Retrieval Conference \footnote{\url{http://trec.nist.gov/}}-, TAC - Text Analysis Conference \footnote{\url{http://www.nist.gov/tac/}}) - y, a su vez, estas distintas competencias ofrecen distintos llamados a competencias. Elegimos resolver una tarea de la competencia Clef '07  por varias razones (Ver \cite{GuidelineClef07} y \cite{OverviewClef07} para un detalle exhaustivo de la conferencia en cuestión). La razón principal fue la pertinencia de la tarea a evaluar al scope de esta tesis. Muchas competencias exigen un grado de complejidad que excede por mucho lo que puede alcanzarse en el tiempo estimado de una tesis de licenciatura y, si bien tuvimos que recortar ciertos aspectos de las tareas a fin de implementar este proyecto en tiempo y forma, estos aspectos fueron pocos.
Otra razón fue la disponibilidad y el atractivo de la base de conocimiento para estos ejercicios: utilizan imágenes de wikipedia.
La competencia del '07 ofrece dos tipos de tareas:
\begin{itemize}
\item Monolingual: donde el idioma de la pregunta y el idioma de la fuente de información son el mismo.
\item Cross-lingual: donde el idioma de la pregunta y el idioma de la fuente de información difieren.
\end{itemize}
Las tareas consideran los siguientes idiomas: inglés, búlgaro, alemán, español, italiano, francés, holandés, rumano y portugués. Por su parte, algunos problemas utilizan fuentes de datos privados de la competencia y otros utilizan como fuente las distintas wikipedias. De los problemas que utilizan wikipedia, implementamos un sistema que responde las preguntas en español, mono-idioma, es decir, ejercicios con preguntas formuladas en español que se responden en base a la wikipedia en español e hicimos lo mismo para el portugués.
A su vez, implementamos esta misma solución para el inglés, dado que estaban disponibles las preguntas y señalados los links a las imágenes de wikipedia en inglés, pero no fue posible evaluar sus resultados debido a que las respuestas esperadas no estaban disponibles online y no obtuvimos respuesta de los organizadores de la competencia. El uso estructural de la librería freeling permite la implementación de soluciones para otros idiomas mediante el set-up del corpus en el idioma y una pequeña configuración.
Los ejercicios elegidos constan de 200 preguntas agrupadas. Los grupos de preguntas refieren a un tema, inferible a partir de la primer pregunta.
Por ejemplo, el primer grupo de preguntas es:
\begin{itemize}
\item ¿En qué colegio estudia Harry Potter?
\item ¿Cuál es el lema del colegio?
\item ¿En qué casas está dividido?
\item ¿Quién es el director del colegio?
\end{itemize}
Es decir, para cada grupo se debe inferir el \dq{tema} en la primer pregunta para arrastrarlo a la hora de responder las siguientes. Más allá de esta particularidad, las preguntas son preguntas simples. Más adelante haremos un análisis de las mismas con más detalle.



\section{Estructura de la tesis}
\label{sec:estructura}

La tesis se articula de la siguiente manera: en la Introducción (\ref{chap:1}), que estamos concluyendo en estos párrafos, realizamos en primer lugar una introducción mínima al área del question answering (\ref{sec:que-es-qa}), en segundo lugar mencionamos los alcances de esta tesis (\ref{sec:proyecto}), y en tercer lugar, aquí, damos una estructura general de la tesis (\ref{sec:estructura}).

En el siguiente capítulo (\ref{chap:teorico}) recorreremos los conceptos generales de algunas áreas en las que se apoya question answering, repasando primero la terminología básica (\ref{sec:terminologia}) para pasar luego a comentar estructuras típicas de recuperación de la información (\ref{sec:information-retrieval}) y, finalmente, de procesamiento de lenguajes (\ref{sec:nlp}) aplicado a problemas de question aswering.

En el capítulo III (\ref{chap:estado-de-arte}) pasamos revista general del estado de arte del área. Primero realizamos una introducción general (\ref{sec:intro-general-qa}), considerando la historia de la disciplina, las competencias en las que la investigación se nucleó (\ref{subsec:historia}) y las métricas utilizadas por estas competencias para evaluar a los competidores (\ref{subsec:metricas}), luego hacemos un recorrido de diferentes enfoques, considerando el estado de arte académico para dominio abierto (\ref{subsec:open-domain}) y dominio cerrado (\ref{subsec:closed-domain}) y finalmente, el comercial, reseñando el funcionamiento de Watson de IBM (\ref{subsec:ibm-watson}).

En los siguientes capítulos presentamos los modelos implementados:

En el IV (\ref{chap:4}) comentamos el modelo de dominio cerrado para la base de datos World, presentando primero la base de datos (\ref{sec:popescu-db}) y luego la implementación de nuestro sistema (\ref{sec:popescu-implementacion}), pasando revista de su código (\ref{subsec:popescu-codigo}), dando ejemplos (\ref{subsec:popescu-ejemplos}) y presentando conclusiones, limitaciones y trabajo futuro(\ref{subsec:popescu-cierre}).

En el V (\ref{chap:5}) presentamos el modelo de dominio abierto para las wikipedias de diferentes idiomas, presentando en primer lugar los problemas seleccionados de la competencia Clef '07 (\ref{sec:ejercicio-de-clef}), en segundo lugar la implementación de nuestro sistema (\ref{sec:sistema}), presentando el sistema baseline de basado en Qanus (\ref{subsec:baseline}), las adaptaciones multilingües y demás mejoras realizadas (\ref{subsec:modificaciones}), las diferentes corridas que realizamos para evaluarlo (\ref{sec:eval}) y, finalmente, las conclusiones, los límites y potenciales mejoras del sistema (\ref{sec:clef-cierre}).
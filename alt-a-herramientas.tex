\appendix
\chapter{Herramientas}
\label{chap:herramientas}

\section{Stanford Question Classifier}
\label{sec:stanford-qc}
El Question Classifier de Stanford \cite{QC2} es un sistema de clasificación de preguntas basado en machine learning que utiliza la arquitectura de aprendizaje SNoW. Es un sistema jerárquico guiado por una semántica de dos niveles que permite la clasificación en categorías granulares. Según los tests de los autores, el proceso logra una efectividad del 90\% sobre 50 diferentes clases finales, utilizando features sintácticos y semánticos. 

El clasificador dispone de una taxonomía de dos capas basada en los tipos de respuesta típicos de la TREC. Las clases superiores son seis: abreviatura, entidad, descripción, humano, lugar y numérico; divididas a su vez en 50 subclases más finas no solapadas (50 en total). La motivación de la existencia de clases superiores es doble: por un lado, ellas preservan compatibilidad y coherencia con las clases usadas en trabajos previos de clasificación de preguntas. Por otro, se esperaba obtener mejoras de performance aplicando dos fases de clasificación, pero esta intención no se corroboró en la experiencia

La tabla siguiente muestra la taxonomía de clasificación dividida en clases y subclases \footnote{Puede encontrarse en esta url: \url{http://cogcomp.cs.illinois.edu/Data/QA/QC/definition.html} y está explicada en \cite{QC2} y \cite{QC3}} 


\begin{center}
\begin{longtable}{| l | l |}
\hline
Clase y subclase & Definición \\ \hline
ABBREVIATION &  abbreviation \\ \hline 
  abb & abbreviation\\ \hline 
  exp & expression abbreviated\\ \hline 
ENTITY  & entities\\ \hline 
  animal  & animals\\ \hline 
  body & organs of body\\ \hline 
  color & colors\\ \hline 
  creative & inventions, books and other creative pieces\\ \hline 
  currency & currency names\\ \hline 
  dis.med. & diseases and medicine\\ \hline 
  event & events\\ \hline 
  food & food\\ \hline 
  instrument & musical instrument\\ \hline 
  lang & languages\\ \hline 
  letter & letters like a-z\\ \hline 
  other & other entities\\ \hline 
  plant & plants\\ \hline 
  product & products\\ \hline 
  religion  & religions\\ \hline 
  sport & sports\\ \hline 
  substance & elements and substances\\ \hline 
  symbol & symbols and signs\\ \hline 
  technique & techniques and methods\\ \hline 
  term  & equivalent terms\\ \hline 
  vehicle & vehicles\\ \hline 
  word & words with a special property\\ \hline 
DESCRIPTION & description and abstract concepts\\ \hline 
  definition & definition of sth.\\ \hline 
  description & description of sth.\\ \hline 
  manner & manner of an action\\ \hline 
  reason & reasons\\ \hline 
%\end{tabular}
%\begin{tabular}{| l | l |}
%\hline
HUMAN & human beings\\ \hline 
  group & a group or organization of persons\\ \hline 
  ind & an individual\\ \hline 
  title & title of a person\\ \hline 
  description & description of a person\\ \hline 
LOCATION & locations\\ \hline 
  city & cities\\ \hline 
  country & countries\\ \hline 
  mountain & mountains\\ \hline 
  other & other locations\\ \hline 
  state & states\\ \hline 
NUMERIC & numeric values\\ \hline 
  code  & postcodes or other codes\\ \hline 
  count & number of sth.\\ \hline 
  date  & dates\\ \hline 
  distance &  linear measures\\ \hline 
  money & prices\\ \hline 
  order & ranks\\ \hline 
  other & other numbers\\ \hline 
  period  & the lasting time of sth.\\ \hline 
  percent & fractions\\ \hline 
  speed & speed\\ \hline 
  temp & temperature\\ \hline 
  size & size, area and volume\\ \hline 
  weight & weight\\ \hline 
\end{longtable}
\end{center}


Uno de los principales problemas que tuvieron los autores al enfrentarse a clases tan específicas fue la ambigüedad instrínseca de ciertas preguntas. Los siguientes ejemplos de este problema están tomados de \cite{QC2}:
\begin{itemize}
\item What is bipolar disorder? (¿Qué es el desorden bipolar?)
\item What do bats eat? (¿Qué comen los murciélagos?)
\end{itemize}
La primer pregunta puede pertenecer a la clase \textit{definición} o bien a la clase \textit{enfermedad / medicina} y la segunda a \textit{comida}, \textit{planta} o \textit{animal}. Para abordar este problema, el clasificador asigna diferentes clases ponderadas y no una única clase.

Los dos niveles de clasificación están implementados como dos clasificadores simples en secuencia, ambos utilizando el algoritmo Winnow de SNoW. El primero etiqueta la pregunta en función de las clases más generales y el segundo asigna las clases más finas (dentro de las suclases determinadas por la clase del primero). 
El modelo para el algoritmo de aprendizaje representa las preguntas como listas de características (\textit{features}) tanto sintácticos como semánticos. Los features utilizando son, en total, más de 200.000, siendo casi todos combinaciones complejas de un set acotado de features simples basados en palabras, pos tags, chunks (componentes constitucionales de la oración), chunks principales (por ejemplo: componente nominal principal), entidades nombradas y palabras semánticamente relacionadas a ciertas clases. De estos seis tipos de features primitivos, tres son sintácticos (pos tags, chunks, chunks principales) mientras que otros son semánticos (named entities, palabras relacionadas).

El clasificador, al igual que el resto de las herramientas de nlp de Stanford, está implementado en java.


\section{Stanford POS \& NER Taggers}
\label{sec:stanford-pos}
\label{sec:stanford-both}
El POS tagger de Stanford es un algortimo basado en entropía máxima (\textit{maximum entropy}), implementado en java originalmente en \cite{POS2}, con algunos agregados y mejoras técnicas realizadas en \cite{POS1}. Al descargar el paquete, también está disponible uno más complejo con soporte para los idiomas árabe, chino y alemán. En este trabajo utilizamos el paquete sencillo, que consta de dos modelos entrenados para inglés, usando, como señalamos en \allref{subsec:pos} el tagset de Penn Treebank. 


Por su parte, el NER tagger de Stanford, es una implementación general de 

It comes with well-engineered feature extractors for Named Entity Recognition, and many options for defining feature extractors. Included with the download are good named entity recognizers for English, particularly for the 3 classes (PERSON, ORGANIZATION, LOCATION), and we also make available on this page various other models for different languages and circumstances, including models trained on just the CoNLL 2003 English training data. The distributional similarity features in some models improve performance but the models require considerably more memory.

Stanford NER is also known as CRFClassifier. The software provides a general implementation of (arbitrary order) linear chain Conditional Random Field (CRF) sequence models. That is, by training your own models, you can actually use this code to build sequence models for any task. (CRF models were pioneered by Lafferty, McCallum, and Pereira (2001); see Sutton and McCallum (2006) or Sutton and McCallum (2010) for more comprehensible introductions.)

The CRF code is by Jenny Finkel. The feature extractors are by Dan Klein, Christopher Manning, and Jenny Finkel. Much of the documentation and usability is due to Anna Rafferty. The CRF sequence models provided here do not precisely correspond to any published paper, but the correct paper to cite for the software is:

Jenny Rose Finkel, Trond Grenager, and Christopher Manning. 2005. Incorporating Non-local Information into Information Extraction Systems by Gibbs Sampling. Proceedings of the 43nd Annual Meeting of the Association for Computational Linguistics (ACL 2005), pp. 363-370. http://nlp.stanford.edu/~manning/papers/gibbscrf3.pdf
The software provided here is similar to the baseline local+Viterbi model in that paper, but adds new distributional similarity based features (in the -distSim classifiers). The big models were trained on a mixture of CoNLL, MUC-6, MUC-7 and ACE named entity corpora, and as a result the models are fairly robust across domains.
Named Entity Recognition Stanford Named Entity Recognizer (Finkel, 
Grenager and Manning 2005) 

Finkel, Jenny Rose, Trond Grenager, and Christopher Manning. "Incorporating 
Non-local Information into Information Extraction Systems by Gibbs Sampling." 
roceedings of the 43nd Annual Meeting of the Association for Computational 
Linguistics. 2005. 



\section{Freeling}
\label{sec:freeling}
Freeling es una librería de c\'odigo abierto que provee distintas herramientas de 
procesamiento de lenguaje natural, desarrollada y mantenida por el Centre de Tecnologies 
i Aplicacions del Llenguatge i la Parla (TALP) de la Universidad Politécnica de Catalu\~na (UPC). 
Freeling está dise\~nado para ser usada como una librería externa y ofrece un API en distintos lenguajes
de programaci\'on. Su principal virtud es ser multilingüe, esto es, los diferentes analizadores que provee funcionan 
para un conjunto bastante amplio de idiomas. La última versi\'on a la fecha (3.1) soporta los siguientes idiomas:

\begin{itemize}
\item Asturian (as)
\item Catalan (ca) 
\item English (en)
\item French (fr) 
\item Galician (gl)
\item Italian (it)
\item Portuguese (pt)
\item Russian (ru)
\item Slovene (sl)
\item Spanish (es)
\item Welsh (cy)
\end{itemize}

Cabe destacar que no todos los m\'odulos soportan todos los idiomas. Sin embargo, dado que el proyecto está radicado en Espa\~na,
los idiomas necesarios para los fines de nuestro trabajo (espa\~nol e inglés), soportan todos los m\'odulos disponibles
en la librería.
Freeling 3.1 ofrece los siguientes analizadores lingüisticos:

\begin{itemize}
\item Detecci\'on de idioma
\item Tokenizer
\item Sentence splitting,
\item Análisis morfol\'ogico
\item NER y NEC (Detecci\'on y Clasificaci\'on de Entidades Nombradas)
\item Reconocimiento de fechas, números, magnitudes físicas, monedas
\item Codificaci\'on fonética
\item POS tagging, 
\item Shallow parsing
\item Dependency parsing
\item Wordnet-based sense annotation
\item Word Sense Disambiguation
\item Coreference resolution
\end{itemize}


\subsection{Módulos de Freeling}
\label{subsec:freeling-pos}
\label{subsec:freeling-mods}
El módulo de POS tagging de Freeling dispone de dos tagger. El primero, $hmm\_tagger$, es un algoritmo clásico que utiliza Hidden Markov Models (HMMs) con tri-gramas. La descripción del algoritmo de tagging basado en HMM se encuentra con detalle en  en \cite{POS0}. Por otro lado, el módulo incorpora un método llamado $relax\_tagger$ que permite la creación de un sistema híbrido que permite reglas escritas a mano y modelos estadísticos. 


\section{Lang Detect de Cybozu Labs}
\label{sec:cybozu}

Librería de Cybozu Labs - una compañía japonesa -, implementado en Java y liberado bajo Apache License 2.0. En la práctica, este paquete dio excelentes resultados. Soporta 53 idiomas con \%99 de precisión para todos ellos (según sus tests). El detector se basa en perfiles de idiomas generados a partir de las distintas wikipedias y detecta el idioma de los textos usando un filtro bayesiano ingenuo (\textit{naive bayesian}).
El código está disponible, actualmente, en google-code (El link está en la sección de bibliografía \cite{nakatani2010langdetect})


\section{Apache Lucene}
\label{sec:lucene}
Lucene es una librería de information retrieval, de c\'odigo abierto, escrita en Java y distribuida 
bajo la licencia Apache Software License por la Apache Software Foundation. No está pensada para
usuarios finales sino para ser integrada dentro de proyectos informáticos, resolviendo
la parte de bajo nivel y brindando servicios a través de un API en diferentes lenguajes de programaci\'on.
Su core es un índice invertido como el que describimos anteriormente. La implementaci\'on de un sistema
que utiliza Lucene consta de dos pasos separados:
\begin{itemize}
\item La \textbf{creaci\'on} del índice, es por lo general un proceso offline en el cual 
se incorporan distintas fuentes de informaci\'on al índice 
\item La \textbf{búsqueda} de documentos en el índice creado en el paso anterior, a partir de una query 
ingresada por el usuario final. Este proceso se incorpora dentro del flujo `online' del sistema.
El resultado de esta búsqueda es una lista de documentos rankeados con un cierto puntaje. 
\end{itemize}

Es importante señalar que si bien el proceso de creaci\'on del índice suele estar desacoplado del resto 
del sistema, las fuentes de informaci\'on no tiene por que ser `offline' en el sentido de ser documentos
en un disco local. De hecho, Nutch, otro proyecto de c\'odigo abierto de la Apache Software Foundation es 
un motor de búsqueda web basado en Lucene que incorpora un crawler para indexar sitios web. Lucene soporta 
cualquier fuente de informaci\'on que pueda convertirse en texto mediante algoritmia.
\newline
Los conceptos fundamentales de Lucene son: índice, documento, campo, término y query.
\begin{itemize}
\item Un índice contiene un conjunto de documentos
\item Un documento es un conjunto de campos
\item Un campo es el nombre de una secuencia de términos
\item Un término es un token (una palabra)
\item Una query es una lista de términos conectados con distintos operados l\'ogicos
\end{itemize}

\bigskip
[[Dar ejemplos de una query]]
\bigskip


\chapter{Comparadores}
\label{sec:comparadores}

Dada la frecuencia en la que resultaba necesario comparar dos string,
decidimos reificar la operación de comparación como una familia de
clases que implementan Comparadores. 


Gracias a esto se hace posible cambiar las nociones de igualdad o
similaridad en un módulo completo del sistema o en alguna clase
simplemente configurando otro comparador como parámetro. La
interfaz permite a los distintos comparadores tomar valores binarios
así como también valores entre 0 y 1. A su vez, esta considerada la
posibilidad de configurar un umbral (threshold) a partir del cual
redondear un valor entre 0 y 1 a un valor binario. Otro factor que tuvo
mucha utilidad fue la capacidad de anidar comparadores. 
El concepto de comparador incluye cualquier operación que tome dos
strings y genere un resultado booleano o analógico. Es decir, es posible 
incorporar análisis lingüísticos o queries a la base de datos en ellos.
También se incorpora la posibilidad de ignorar o no ignorar la diferencia de mayúsculas, 
y obviar o no las tildes y otros signos problemáticos. Los comparadores sirven,
en general, para comparar tanto strings representando palabras como
string representado listas de palabras (oraciones o textos). Algunos,
en particular, sólo sirven para este segundo caso. Los comparadores
que finalmente utilizamos en esta tesis son los siguientes.

\begin{center}
\begin{tabular}{| l | p {8cm} |}
\hline
\multicolumn{2}{|c|}{Comparadores de Strings} \\ \hline
Nombre & Descripción\\ \hline 
Equal & Compara por igualdad estricta \\ \hline 
EqualNoPunct &  Compara por igualdad, eliminando signos de
puntuación y normalizando acentos y otras posibles diferencias que no
deberían tenerse en cuenta. \\ \hline 
Contains & Verifica si un string contiene a otro. Puede usar Equal o EqualNoPunct \\ \hline 
EditDistance & Verifica cuan similares son dos string contando la mínima cantidad de operaciones requeridas para transformar un string en el otro \\ \hline 
\end{tabular}
\end{center}

Los siguiente comparadores son algoritmos fueron adaptados a partir de
los Scorers del proyecto Qanus. Todos devuelven valores reales entre
0 y 1 y sirven para comparar secuencias de tokens (y no sólo palabras). Estos
comparadores, al igual que Contains, no son simétricos. Para
distinguir, llamaremos primer string al buscado y segundo string a
aquel en el cual se busca el primero. 

\begin{center}
\begin{tabular}{| l | l | p {8cm} |}
\hline
\multicolumn{3}{|c|}{Comparadores de Secuencias de Tokens} \\ \hline
Abreviatura & Nombre &  Descripción\\ \hline 
Freq & Frequencia & Computa la cantidad de veces que los tokens del primer
string ocurren en el segundo string. Esta suma se divide por la
longitud del segundo string, dando un valor entre 0 y 1. \\ \hline 
Covr & Cobertura &  Computa cuantos tokens del primer string aparecen al
menos una vez en el segundo, y divide esta suma por el total de tokens
del \textit{primer} string.\\ \hline 
Prox & Proximidad &  Computa la distancia entre dos strings en un tercero. Ver abajo.   \\ \hline 
Span & Distancia entre tokens & Computa la distancia media entre términos del primer string en el segundo. Ver abajo. \\ \hline
\end{tabular}
\end{center}

Vamos a explicar los algoritmos de $Prox$ y $Span$ ya que no son triviales. \newline
$Prox$ toma dos strings a buscar en un tercero. Busca ambos en el tercero y computa la distancia en tokens entre ellos.
Esta distancia se calcula como la distancia entre el centro de ambos strings.
Por ejemplo, para los strings de búsqueda \dq{Argentina es un país americano} y \dq{independizado en 1810} sobre el texto \dq{Argentina es un país americano, originalmente una colonia española, independizado en 1810} se considera la distancia entre \sq{un} y \sq{en} (por ser los tokens \sq{intermedios} de ambos strings
de búsqueda. La distancia entre ambos, en el tercer string, es 7. Esta distancia se divide por la longitud en tokens del string en el que se buscan (12), dando un resultado de 0.58. Un score cercano a 1 denota que los dos string están cercanos uno al otro en el tercer string. \newline
Por su parte, $Span$ tiene un concepto similar, pero funciona sobre un solo string de búsqueda, considerando sus tokens. Los distintos tokens buscados ocurren en ciertas posiciones. $Span$ considera la distancia entre las posiciones de los tokens más distantes, dividiendo el total de tokens encontrados por este valor.
Un score cercano a 1 significa que los términos del string buscado están cerca en el string en el que se buscan.
Por ejemplo, suponiendo los siguientes matchs de tokens (denotados por una X): \newline
..... X ..... X ..... X ...... \newline
......a ...... b ...... c ...... \newline

El score de $Span$ estaría dado por \#total de tokens encontrados /  {\textbar}c-a{\textbar}.

\chapter{Resultados para inglés simple}
\label{chap:resultados-ingles-simple}

\begin{longtable}{ | p {4cm}| l | p {4cm} |}
    \hline
    Idioma & Fecha & Tamaño\\ \hline
In what school does Harry Potter study? & Transsexual  &  Colegio Hogwarts de Magia y Hechicería \\ \hline
Which is the school motto? & Moon Unit  &  "Draco Dormiens Nunquam Titillandus" \\ \hline
In how many houses is it divided? & Null  &  * Gryffindor    * Slytherin    * Ravenclaw    * Hufflepuff \\ \hline
Who is the school principal? & George Hearst  &  Albus Percival Wulfric Brian Dumbledore \\ \hline
What do swordfishes look like? & Null  &  son grandes peces predadores altamente migratorios, caracterizados por su pico largo y aplanado \\ \hline
Which is the IGFA record for this kind of fish? & Brato  &  536 kg \\ \hline
When was Amintore Fanfani born? & Mae West  &  6 de febrero de 1908 \\ \hline
And where was he born? & Birmingham  &  Provincia de Arezzo, Italia \\ \hline
Who was Le Corbusier? & Null  &  arquitecto
 francés \\ \hline
What was his real name? & Dr Benjamin  &  Charles-Edouard
 Jeanneret-Gris \\ \hline
Who was Flaubert? & Null  &  novelista francés \\ \hline
In what year did he publish "Bouvard et Pécuchet"? & 1942  &  1881 \\ \hline
Enumerate all his works. & Null  &  * Madame Bovary (1856)    * Salambó (1862)    * La educación sentimental (1869)    * La tentación de San Antonio (1874)    * Tres cuentos (1877)    * Bouvard y Pécuchet \\ \hline
What was the Velvet Revolution? & Null  &  el movimiento pacífico por el cual el partido comunista de Checoslovaquia perdió el monopolio del poder y se volvió a la democracia \\ \hline
What was the Carnation Revolution? & Null  &  levantamiento militar del 25 de abril de 1974 que provocó la caída en Portugal de la dictadura salazarista \\ \hline
Who was António de Oliveira Salazar? & Null  &  dictador portugués \\ \hline
Who composed the song "Grândola, Vila Morena"? & McCartney  &  José Afonso \\ \hline
Where was Bender Bending Rodriguez manufactured? & California  &  México \\ \hline
Who is Philip J. Fry? & Null  &  es el protagonista de Futurama \\ \hline
Who was Hermann Emil Fischer? & Null  &  químico alemán, receptor del Premio Nobel de Química \\ \hline
What prize did he receive in 1902? & distance telephone call  &  Premio Nobel de Química \\ \hline
Who was awarded the Nobel Prize in Literature that year? & Lorentz  &  Theodor Mommsen \\ \hline
Who was Bertha von Suttner? & Null  &  pacifista y escritora austríaca \\ \hline
Who was Marco Pantani? & Null  &  ciclista italiano \\ \hline
Which was his nickname? & Milne  &  Il Pirata \\ \hline
Who was Juan Manuel Fangio? & Null  &  pentacampeón del mundo de Fórmula Uno \\ \hline
What is an obsidian? & Null  &  roca del tipo ígnea extrusiva perteneciente al grupo de los silicatos \\ \hline
What is a "macana"? & Null  &  mazas de madera que utilizaban los guerreros precolombinos \\ \hline
What is a black hole? & Null  &  una estrella o un grupo de estrellas con tal
 densidad que absorbe totalmente toda materia o energía que esté en su
 campo de gravedad \\ \hline
What is virtual reality? & Null  &  un sistema o interfaz informático que genera entornos sintéticos en tiempo real \\ \hline
What is a "pasodoble"? & Null  &  una variedad musical dentro de la forma marcha y posteriormente un estilo de baile \\ \hline
What is a tarantella? & Null  &  Baile popular del sur de Italia \\ \hline
What is Noah's Ark? & Null  &  una embarcación construida por Noé durante el Diluvio Universal \\ \hline
What is a tachograph? & Null  &  un dispositivo electrónico que registra diversos sucesos originados en un vehículo durante su conducción \\ \hline
What is an odometer? & Null  &  un dispositivo que indica la distancia recorrida en un viaje por automóvil u otro vehículo \\ \hline
What is a kinnor? & Null  &  una lira hebrea portátil de 5 a 9 cuerdas \\ \hline
What is INTASAT? & Null  &  el primer satélite artificial científico español \\ \hline
What was the Rainbow Warrior? & Null  &  navío de la
 organización ecologista Greenpeace \\ \hline
Who sank it in 1985? & Al MacInnis  &  los servicios secretos franceses \\ \hline
What is a minesweeper used for? & Null  &  la identificación y destrucción de minas marinas \\ \hline
What is INTELSAT? & Null  &  una red de satélites de comunicaciones que cubre el mundo entero \\ \hline
What is a Non-Governmental Organization? & Null  &  una entidad de carácter privado, con fines y objetivos definidos por sus integrantes, creada independientemente de los gobiernos locales, regionales y nacionales, así como también de los organismos internacionales \\ \hline
What was the Gestapo? & Null  &  la policía
 política del régimen nazi \\ \hline
How many members did it have during the Second World War? & 1976  &  45.000 \\ \hline
What is the International Organization for Standardization? & Null  &  organización internacional no gubernamental, compuesta por representantes de los organismos de normalización (ONs) nacionales, que produce normas internacionales industriales y comerciales. \\ \hline
What was The Black Hand? & Null  &  una supuesta organización anarquista secreta y violenta que actuó en Andalucía a finales del siglo XIX \\ \hline
What is the purpose of the World Meteorological Organization? & Null  &  asegurar y facilitar la cooperación entre los servicios meteorológicos nacionales, promover y unificar los instrumentos de medida y los métodos de observación. \\ \hline
What is UNIDO in charge of? & Null  &  de promover y acelerar la industrialización en los países en desarrollo \\ \hline
What does Industrial Light and Magic do? & Null  &  a producir efectos visuales y gráficos generados por ordenador para películas \\ \hline
In what country is Çatalhöyük located? & South Africa  &  Turquía \\ \hline
Into which sea does the Orentes river flow? & Qinghai Province  &  Mar Mediterráneo \\ \hline
In which country is Hadrian's Wall? & Concordia  &  Gran Bretaña \\ \hline
At what place does Homer Simpson work? & Wikicities  &  en la planta de energía nuclear de Springfield \\ \hline
In which country is Minas Gerais located? & {\color{red}Rio de Janeiro}  &  Brasil \\ \hline
Where is ensaymada considered traditional? & Lima  &  Mallorca \\ \hline
In what city was Natalie Hershlag born? & Votkinsk  &  Jerusalén \\ \hline
Who is the mayor of that city? & Joseph B. From  &  Ehud Olmert \\ \hline
Who designed the Zilog Z80 processor? & Gerald Scarfe  &  Federico Faggin \\ \hline
How many bits did this processor have? & 6  &  8 bits \\ \hline
What are the names of Luke Skywalker's parents? & Null  &  Padmé Naberrie, Anakin Skywalker \\ \hline
Which countries joined the European Union on 1st January of 1995? & Null  &  Austria, Finlandia y Suecia \\ \hline
Who is the Queen of Australia? & Paul Ramadier  &  Isabel II \\ \hline
Who was Pope when the Council of Clermont was held? & Opus Dei  &  Urbano II \\ \hline
What actor plays the Chewbacca role? & Peter Pan  &  Peter Mayhew \\ \hline
What's the name of the Greek god of medicine? & Deimos  &  Asclepio \\ \hline
Who is the director of Nosferatu? & Tove Jansson  &  F.W. Murnau \\ \hline
And who played the main character? & James T. Kirk  &  Max Schreck \\ \hline
Who is the general secretary of UGT? & Pedro Calderon de la Barca  &  Cándido Méndez \\ \hline
When was the terrorist attack on the AMIA? & Suicide bombing  &  18 de julio de 1994 \\ \hline
How many people died on the attack? & 1914  &  85 personas \\ \hline
When did Federica Montseny die? & Die  &  14 de enero de 1994 \\ \hline
Which day was the assault to the Davidians Ranch in Waco? & George W. Bush  &  19 de abril de 1993 \\ \hline
In which year did the Real Zaragoza win the European Cup Winners' Cup? & 1942  &  1995 \\ \hline
In what year was the AVE Madrid-Seville inaugurated? & 1492  &  1992 \\ \hline
In which date did El Corte Inglés purchase Galerías Preciados? & MediaWiki:Variantname-sr-el  &  24 de noviembre de 1995 \\ \hline
What day was Rajiv Gandhi killed? & Indira Gandhi  &  21 de mayo de 1.991 \\ \hline
Which four gods can be represented on a canopic jar? & Null  &  Imset, Hapi, Kebehsenuf, Duamutef \\ \hline
Who used to manufacture Windows 95? & NA  &  Microsoft \\ \hline
Which company does Steve Jobs chair? & Hamilton  &  Apple Computer \\ \hline
What was the name of the GRD secret service? & Francisco de Almeida  &  Stasi \\ \hline
Of what organization is David Stern commissioner? & Wikipedia  &  NBA \\ \hline
Which radio station broadcasts Howard Stern's Show? & England (  &  Sirius \\ \hline
Which school was founded by Walter Gropius? & Dr Un Yong Kim  &  Instituto de  Artes y
 Oficios Bauhaus \\ \hline
What order did Saint Thomas Aquinas belong to? & Opus Dei  &  Dominicos \\ \hline
What order was founded by Francis of Assisi? & People  &  Franciscanos \\ \hline
Which military forces did Rommel lead? & Rome  &  Afrika  Korps \\ \hline
What company has Bibendum as mascot? & Guinness  &  Michelin \\ \hline
What was the name of Cousteau's ship? & " Nobody  &  Calypso \\ \hline
What sort of milk is used for making mozzarella? & milk  &  leche de búfala \\ \hline
What is the name of King Arthur's sword? & Elohim  &  Excalibur \\ \hline
Which tree bears acorns? & tree  &  encinas \\ \hline
From what plant are tigernuts obtained? & oils  &  juncia almendrada \\ \hline
What material is a Moai made of? & something  &  roca volcánica \\ \hline
What does the bearded vulture eat? & church  &  médula \\ \hline
What was the name of the atomic bomb dropped on Nagasaki? & name  &  Fat Man \\ \hline
Which was the first book printed by Gutenberg? & Document   &  la Biblia \\ \hline
What form does the entrance of Louvre Museum have? & June  &  pirámide \\ \hline
What animals pull Cybele's chariot? & Cattle  &  leones \\ \hline
What does Mohs' Scale measure? & explanation  &  la dureza de una sustancia \\ \hline
What liquor is made from blackthorn? & liquors  &  pacharán \\ \hline
In what museum is Las Meninas of Velázquez? & Paris  &  Museo del Prado \\ \hline
What scandal caused Nixon's resignation? & August  &  Watergate \\ \hline
What does Leica manufacture? & government  &  instrumentos ópticos de precisión \\ \hline
On what side of a boat is port? & nuisance  &  lado izquierdo mirando hacia proa \\ \hline
What are the names of the balls used in Quidditch? & Null  &  quaffle, bludger y snitch \\ \hline
How many people died in the sinking of the USS Maine? & 60,000  &  266 \\ \hline
How many athletes participated in the Olympic Games of Barcelona? & 6  &  9.364 atletas \\ \hline
How many inhabitants does Lithuania have? & 5  &  3,7 millones \\ \hline
How many seats does the parliament of Iceland have? & 15  &  63 \\ \hline
How many years will the global oil reserves last? & 1731  &  43 años \\ \hline
How many times has Alain Prost been World Champion? & 25  &  cuatro \\ \hline
How much horsepower do the turbines of the USS Nimitz have? & one  &  260.000 HP \\ \hline
How many goals were scored overall at the 1982 Football World Cup? & 500  &  146 \\ \hline
How many basements does the Picasso Tower have? & {\color{red}5}  &  5 \\ \hline
How many Goya Awards did "Torrente: El brazo tonto de la ley" win? & 1934  &  dos \\ \hline
How old is the sun? & 1899  &  4500 millones de años \\ \hline
Which is the running time of the movie "The Ninth Gate"? & Paige  &  132 minutos \\ \hline
At what age did Alfred Hitchcock die? & Two  &  80 años \\ \hline
What is the span of the Airbus 380? & Airbus A380  &  79,75 metros \\ \hline
How many passengers did it transport on 4 September 2006? & 1976  &  474 \\ \hline
How much does each unit cost? & one  &  250 millones de EUR \\ \hline
Which is the carrying capacity of the Antonov An-124? & Tennessee  &  130 toneladas \\ \hline
Who are the founding members of the Star Alliance? & Null  &  Air Canada, Lufthansa, SAS Scandinavian Airlines, Thai Airways International, United Airlines \\ \hline
In what country is náhuatl spoken? & Indonesia  &  México \\ \hline
How many curves does the Circuit de Monaco have? & 5  &  18 \\ \hline
Who was the first woman to go in space? & Garfunkel  &  Valentina Vladimírovna Tereshkova \\ \hline
Who has directed "The Day of the Beast"? & Peter Jackson  &  Álex de la Iglesia \\ \hline
Who inaugurated the Temple of Debod in Madrid? & Joseph Smith  &  Carlos Arias Navarro \\ \hline
In which year was Torquemada appointed Grand Inquisitor? & 1712  &  1482 \\ \hline
In what year was the Nebrija's Grammar published? & 2004  &  1492 \\ \hline
Where did Rachel Weisz study literature? & Freud  &  Trinity Hall, Cambridge \\ \hline
In which city is the church of La Sagrada Familia by Gaudí located? & Poland  &  Barcelona \\ \hline
How high is Mount Vesuvius? & )  &  1.281 metros \\ \hline
How many kilometers away is Loarre Castle from Huesca? & 420  &  35 kilómetros \\ \hline
Which is the molar mass of Methanol? & Carbon  &  32.04 uma \\ \hline
How high is the Pyramid of the Sun of Teotihuacan? & one  &  65 m \\ \hline
How many falls form the Iguazu Waterfalls? & One  &  275 saltos \\ \hline
How many steps does the Tower of Pisa have? & 27  &  294 escalones \\ \hline
Which are the ingredients of sangría? & Null  &  * Vino tinto.    * Fruta picada o rebanada.    * Un endulzador como la miel.    * Un poco de brandy, triple sec, u otro licor. \\ \hline
Who is in charge of the security of the Vatican City? & Pedro Calderon de la Barca  &  Guardia Suiza \\ \hline
Who gives the Fields Medal? & Karl Marx  &  la Unión Matemática Internacional \\ \hline
For whom does Jack Bauer work? & Louise Caroline Alberta  &  la agencia UAT \\ \hline
What was used as defoliant in the Vietnam War? & {\color{red}Herbicides}  &  Agente Naranja \\ \hline
Which was the codename of the seaborne invasion of Normandy? & York  &  Operación Overlord \\ \hline
From which mold is the penicillin obtained? & one  &  Penicillium notatum \\ \hline
From what mollusc did Phoenicians extract purple dye? & rainbow  &  Murex brandaris \\ \hline
What did Manuel Campello Esclápez discover? & use  &  Dama de Elche \\ \hline
What did Jason and the Argonauts look for? & times  &  vellocino de oro \\ \hline
Who was the US president during the attack on Pearl Harbour? & U. Dryfuss  &  Roosevelt \\ \hline
What did Norsk Hydro use to manufacture during the World War II? & Jews  &  agua pesada \\ \hline
Which is the title of the Nadal Prize's winner in 1994? & Rio Reiser  &  Azul \\ \hline
How many reservists did the Haganah have in 1936? & 6  &  40.000 \\ \hline
How old was Miguel Induráin during the 1985 Tour of Spain? & , 1920/CD  &  20 años \\ \hline
In what year did he win the Tour of Spain? & 1185  &  NIL \\ \hline
How many stages did he win in "Tour de l'Avenir" in 1986? & 1942  &  Dos \\ \hline
What prize was he awarded in 1992? & McEwan  &  Premio Príncipe de Asturias de los Deportes \\ \hline
What actors are the main characters of "The Good, the Bad and the Ugly"? & Null  &  Clint Eastwood, Elli Wallach, Lee Van Cleef \\ \hline
How many people died in Europe during the plague epidemic of the 14th Century? & 1810  &  25 millones \\ \hline
In which country did the 1918 flu pandemic start? & Sarajevo  &  Tíbet \\ \hline
What are the Catholic Monarchs' names? & Null  &  Fernando de Aragón e Isabel I de Castilla \\ \hline
What is the name of the president of Burundi who died in 1994? & God  &  Cyprien Ntaryamira \\ \hline
Which day was the Spanish constitution of 1812 promulgated? & War of 1812  &  19 de marzo \\ \hline
Who was the captain of the "Real Sociedad" between 1974 and 1989? & Monty Python  &  Luis Miguel Arconada Echarri \\ \hline
In what city was the final of the 1996 European Basketball Cup held? & Portugal  &  Vitoria \\ \hline
Which terrorist group committed a terrorist attack during the Munich Olympic Games? & Francisco de Almeida  &  Septiembre Negro \\ \hline
What political party ruled Spain from 28 October 1982 until 3 March 1996? & Crimean War  &  Partido Socialista Obrero Español (PSOE) \\ \hline
What uniform did the King wear when intervened on television after Tejero's coup d'état? & coup d'état  &  Capitán General de los Ejércitos \\ \hline
What did Leonardo Da Vinci draw in 1492? & Leonardo da Vinci  &  Hombre de Vitruvio \\ \hline
What did Antonio Rebollo use to light the Olympic Torch in the Barcelona Olympic Games? & article  &  una flecha encendida \\ \hline
What TV show did Takeshi Kitano present from 1986 to 1989? & Opus Dei  &  Humor Amarillo \\ \hline
What gift was New York given at the commemoration of the centennial of the USA independence? & John Wayne Gacy  &  La Estatua de la Libertad \\ \hline
Which is the distributor of the film "Planet of the Apes" which was premiered in 1968? & Sagan  &  20th Century Fox \\ \hline
What speed did Louis Blériot reach on 25 July 1909? & test tube baby  &  64 Kms/hr \\ \hline
How many kilos of anchovies did the Cantabric fleet catch during 1994? & 220  &  14 millones de kilos \\ \hline
What island did Argentina dispute to the United Kingdom between 2 April and 14 June 1982? & U. China  &  Islas Malvinas \\ \hline
Where did the Mediterranean Games of 1997 take place? & Iraq  &  Bari \\ \hline
Which boat did James Cook command from 1768 to 1769? & Kamehameha  &  HM Endeavour \\ \hline
Who plays the main character in the movie "Tarzan the Ape Man"? & Plankton  &  Johnny Weissmüller \\ \hline
How many world records did he break? & 40  &  67 \\ \hline
Who won the Academy Honorary Award in 2007? & Linda Hunt  &  NIL \\ \hline
Who stole Bianca Castafiore's jewels? & Buffy Summers  &  NIL \\ \hline
In what country is The Swan Lake located? & Virginia  &  NIL \\ \hline
In what year did Captain America die? & 1778  &  NIL \\ \hline
Which is the capital city of Neverland? & Catalonia  &  NIL \\ \hline
What is the name of Bill Gates' wife? & Dorion Sagan  &  Melinda French \\ \hline
In what university was he studying when Microsoft was created? & 1167  &  Universidad de Harvard \\ \hline
What budget did that university have in 2005? & times  &  25.900 millones de dólares \\ \hline
In what year was it founded? & 1903  &  1636 \\ \hline
Which supersonic plane finished its commercial flights in May 2003? & Kaho'olawe  &  El Concorde \\ \hline
How many passengers could this model of plane carry? & 1976  &  188 \\ \hline
Which was the capacity of the Santiago Bernabéu Stadium in the 80s? & Seville  &  90.800 espectadores \\ \hline
Who is the owner of the stadium? & God  &  Real Madrid Club de Fútbol \\ \hline
How much money was spent during its enlargement between 2001 and 2006? & 10  &  127 millones de euros \\ \hline
How many bombs were dropped over Dresden on 14 February 1945? & 1972  &  1.800 bombas explosivas y 136.800 bombas incendiarias \\ \hline
How much did the bombs dropped on this city on 7 October 1944 weigh? & one  &  80 toneladas \\ \hline
Which organization did Alan Turing lead during the Second World War? & Kamakura Period  &  la sección Naval Enigma del Bletchley Park \\ \hline
Who was the Russian Tsar during the Second World War? & Meiji  &  NIL \\ \hline
How many points did Yugoslavia score in the 1990 Basketball World Championship final? & 587  &  92 \\ \hline
Against which country did Spain play the final of basketball at the 1984 Summer Olympics? & Netherlands  &  Estados Unidos \\ \hline
In what year did Spain win the Football World Cup? & 1185  &  NIL \\ \hline

\end{longtable}

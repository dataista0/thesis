\chapter{Grafo de Mitic}
\label{chap:db}

\begin{center}
\begin{table}[H]
\centering
\begin{tabular}{| l | l |}
\hline
\multicolumn{2}{|c|}{Universidades} \\ \hline
Atributo & Descripción \\ \hline
Canónico & Nombre o abreviatura (UBA, Universidad de San Andrés)\\ \hline
Nombre& Nombre completo de la institución\\ \hline
%Carreras & Lista de carreras dictadas {\color{red} HAY UNA. QUE HAGO?} \\ \hline
Dirección & Domicio físico\\ \hline
Página & Página web\\ \hline
\end{tabular}
\caption{Atributos de la colección Universidades}
\label{table:universidades}
\end{table}
\end{center}

\begin{center}
\begin{table}[H]
\centering
\begin{tabular}{| l | l |}
\hline
\multicolumn{2}{|c|}{Investigadores} \\ \hline
Atributo & Descripción \\ \hline
	Nombre y Apellido & \\ \hline
	Fecha Nacimiento & \\ \hline
	E-mail & \\ \hline
  Telefono & \\ \hline
  Titulo & \\ \hline
  Centro & \\ \hline
	Lugares Trabajo & \\ \hline
	Tema & \\ \hline
	Url Personal & \\ \hline
\end{tabular}
\caption{Atributos de la colección Investigadores}
\label{table:investigadores}
\end{table}
\end{center}

\begin{center}
\begin{table}[H]
\centering
\begin{tabular}{|  l | l |}
\hline
\multicolumn{2}{|c|}{Temas} \\ \hline
Atributo & Descripción \\ \hline
Nombre &  automatizacion industrial , telefonia ip , inteligencia artificial\\ \hline 
\end{tabular}
\caption{Atributos de la colección Temas}
\label{table:temas}
\end{table}
\end{center}

\begin{center}
\begin{table}[H]
\centering
\begin{tabular}{| l | l |}
\hline
\multicolumn{2}{|c|}{Empresas} \\ \hline
Atributo & Descripción \\ \hline
Nombre & \\ \hline 
Año Fundación & \\ \hline 
Cantidad Empleados & \\ \hline 
Domicilio & \\ \hline 
Ciudad & \\ \hline 
Provincia & \\ \hline 
ID LinkedIn & \\ \hline 
Telefono & \\ \hline 
Website url & \\ \hline 
Descripcion & \\ \hline 
Especialidades & \\ \hline 
\end{tabular}
\caption{Atributos de la colección Empresas}
\label{table:empresas}
\end{table}
\end{center}

\begin{center}
\begin{table}[H]
\centering
\begin{tabular}{| l | l |}
\hline
\multicolumn{2}{|c|}{Proyectos} \\ \hline
Atributo & Descripción \\ \hline
Titulo & \\ \hline 
Keywords & \\ \hline 
Director & \\ \hline 
Investigadores Nombre & \\ \hline 
Entidad Benefactora & \\ \hline 
Instrumento & \\ \hline 
Descripcion & \\ \hline 
Año & \\ \hline 
Año Hasta & \\ \hline
\end{tabular}
\caption{Atributos de la colección Proyectos}
\label{table:proyectos}
\end{table}
\end{center}

\begin{center}
\begin{table}[H]
\centering
\begin{tabular}{|  l | l |}
\hline
\multicolumn{2}{|c|}{Publicaciones} \\ \hline
Atributo & Descripción \\ \hline
Titulo & \\ \hline 
Keywords & \\ \hline 
Investigadores Nombre & \\ \hline 
Journal & \\ \hline 
Resumen & \\ \hline 
Url Doc  & \\ \hline 
\end{tabular}
\caption{Atributos de la colección Publicaciones}
\label{table:publicaciones}
\end{table}
\end{center}


En nuestro modelo eliminamos varios campos que no tenían datos o tenían muy pocos (menos del 0.02\% del total de documentos en la relación). Ellos son: el universidades.areas\_inv, universidades.carreras, empresas.blog\_rss\_url, empresas.twitter\_id, investigadores.disciplina, investigadores.grado\_obtenido, investigadores.instituto, investigadores.locacion, investigadores.numero\_documento, proyectos.codirector, proyectos.provincia\_proy y publicaciones.publisher.



\chapter{Placeholders}
\section{MyMemory}
MyMemory es la Memoria de Traducción más grande del mundo: 300 millones de segmentos a finales de 2009

Como las MT tradicionales, MyMemory almacena segmentos con sus traducciones, ofreciendo a los traductores correspondencias y concordancias. El proyecto se diferencia de las tecnologías tradicionales por sus ambiciosas dimensiones y por su arquitectura centralizada y basada en la colaboración colectiva. Todos pueden consultar MyMemory o hacer aportaciones a través de Internet; la calidad de las aportaciones será cuidadosamente ponderada.
MyMemory gives you quick access to a large number of translations originating from professional translators, LSPs, customers and multilingual web content. It uses a powerful matching algorithm to provide the best translations available for your source text. MyMemory currently contains 644.377.834 professionally translated segments.

Las memorias de traducción son almacenes compuestos de textos originales en una lengua alineados con su traducción en otras. Esta definición de memorias de traducción coincide literalmente con una de las definiciones más aceptadas de corpus lingüístico de tipo paralelo (Baker, 1995). Por esto se puede decir que las memorias de traducción son corpus paralelos.


Por el momento, no estamos utilizando ningún módulo de traducciones:
todo el enfoque multilingüe está dado por la detección del idioma
de la pregunta y la determinación de distintas herramientas de
análisis según qué idioma sea. Sin embargo, en un momento se
evaluó un enfoque distinto, basado en la traducción. Por ejemplo:
utilizar módulos de procesamiento sólo en inglés y
{\textquotedblleft}normalizar{\textquotedblright} los inputs en otros
idiomas (en principio, en espa\~nol), a este idioma interno, y luego lo
mismo con la generación de respuestas. A pesar de que no es el
enfoque actual, hubo una fase de investigación dentro del dominio de
la traducción, que resultó en un módulo de traducción basado en
MyMemoryAPI.

Intento con google translator y la privatización. ?`La falta de
software de traducción offline? El módulo de mymemory, robado de
algún lugar. El sistema de cobro. 

\section{Citas a Textos trascriptos pero no usados}
\begin{itemize}
\item QC: \cite{QC1}, \cite{QC2} y también \cite{QC3} (y \cite{QC-other})
\item Clef: \cite{GuidelineClef07} y \cite{OverviewClef07} 
\item POS: El manual \cite{POS0} y los dos de Stanford: \cite{POS1} y \cite{POS2}
\item LangDetect: \cite{nakatani2010langdetect}
\item NER: Survey \cite{NER1} y el NER de Stanford: \cite{NER2}
\item Watson: \cite{WATSON1} y \cite{WATSON2}
\item Qanus: \cite{QANUS1}
\item RE: Survey \cite{RE1}, for QA \cite{RE2} y reverb: \cite{RE3}
\item Ephyra: \cite{EPHYRA1}
\item Freeling: \cite{FREELING1} y \cite{FREELING2} (este no impreso)
\item Wordnet para web ir: \cite{WN1} (no leido)
\item Varios de QA: Yago \cite{YAGO-QA1}, sobre una teoria de QA como interfaz a DBs: \cite{QADB1}. Corpus: \cite{TRAIN-QA1}, qall-me: \cite{QALL-ME1}, practical QA: \cite{QAS1}, simple QA: \cite{QAS2} y Surface de Ravishandran: \cite{SURF1}. Introducción a QA: \cite{QA1} y \cite{QA2} y \cite{QA3}
\item Aranea: \cite{ARANEA1} (no leido)
\item Passage retrieval evaluation: \cite{PASSAGE1}
\item Evaluacion de las TREC8 (metrica de \cite{QA3} LASSO): \cite{TREC8}
\item QA survey: \cite{QA-survey}
\end{itemize}

\documentclass{beamer}

\mode<presentation> {
  \usetheme{Warsaw}
  \setbeamercovered{transparent}
}


\usepackage[utf8]{inputenc}
\usepackage[spanish]{babel}
%\usepackage[latin1]{inputenc}

\usepackage{amsmath,amssymb}
\usepackage{graphicx}
\usepackage{fancyvrb}

\usepackage{tikz}
\usetikzlibrary{arrows,shapes}
\usetikzlibrary{shadows}
\usetikzlibrary{shapes.arrows}

\usepackage[3D]{movie15}




\title{Question answering de dominio abierto y de dominio cerrado}

\author{Julián Peller}

\date{Abril 2016} %

\subject{Defensa de tesis}

%\pgfdeclareimage[height=0.8cm]{university-logo}{logo}
%\logo{\includegraphics[width=8mm]{logo_uva.jpg}}



%\beamerdefaultoverlayspecification{<+->}

\begin{document}




\begin{frame}
  \titlepage
\end{frame}


\begin{frame}
  \frametitle{index}
  \tableofcontents[pausesections]
\end{frame}

\section{Introducción}

\subsection{Qué es question answering}
\begin{frame}
        \begin{block}{Question Answering}<3->
            Es el proceso automatizado de generación de respuestas concretas para preguntas formuladas en lenguaje natural.
        \end{block}
        \bigskip

       \begin{alertblock}{Question}<1->
            ¿Quién desarrolló la teoría de la relatividad?
        \end{alertblock}

        \begin{columns}<3->
            \begin{column}{.5\textwidth}
            \end{column}
            \begin{column}{.1\textwidth}
            \begin{tikzpicture}[>=stealth, rotate border/.style={shape border uses incircle, shape border rotate=270}]
                    \node[rotate border=-40, fill=black, minimum height=1.5cm, single arrow, single arrow head extend=.3cm, single arrow head indent=.1cm, inner sep=1.5pt] (arrow) {};
                \end{tikzpicture}
            \end{column}
            \begin{column}{.3\textwidth}
                %Question Answering
            \end{column}
            \begin{column}{.5\textwidth}

            \end{column}
        \end{columns}

        \begin{exampleblock}{Answer}<2->
            Albert Einstein.
        \end{exampleblock}
\end{frame}
\begin{frame}


\begin{block}{}
Pregunta
\end{block}
\medskip
Flecha vertical
\medskip
\begin{block}{}
Respuesta
\end{block}
\bigskip


\end{frame}
\subsubsection*{Diferentes tipos de QA}
\subsection{Overview del proyecto}
\subsubsection*{Herramientas}

\section{Dominio cerrado}
\subsection{Marco de trabajo}
\subsubsection*{Dominio de problemas y enfoque}
\subsubsection*{Modelo teórico de Popescu}
\subsection{Implementación}
\subsubsection*{Limitaciones, trabajo futuro, conclusiones}
\section{Dominio abierto}
\subsection{Marco de trabajo}
\subsubsection*{Dominio de problemas y enfoque}
\subsubsection*{Ejercicio de Clef}
\subsection{Implementación}
\subsubsection*{Experimientación}
\subsubsection*{Limitaciones, trabajo futuro, conclusiones}
\section{Cierre}
  \subsection*{Cierre}
  \subsection*{Preguntas}

\begin{frame}
    \frametitle{Título de la primera diapositiva}
\begin{columns}
 \begin{column}{0.55\textwidth}
\begin{block}{Resultados}
    \begin{itemize}[<+->]
      \item  Primer item
      \item  Segundo item
      \item  Tercer item
    \end{itemize}
    \begin{enumerate}[<+-| alert@+>]
      \item  Primer item
      \item  Segundo item
      \item  Tercer item
    \end{enumerate}
\end{block}
 \end{column} \ \
% \begin{column}{0.40\textwidth}
      %\only<4>{\includegraphics[width=0.8\textwidth]{knuth.jpg}}
      %\only<5>{\includegraphics[width=0.9\textwidth]{forges.jpg}}
   %   \only<6>{\includegraphics[width=0.9\textwidth]{kill-bill-2.jpg}}
% \end{column}
\end{columns}
\end{frame}



\end{document}

\usetheme{default}
\usetheme{JuanLesPins}
\usetheme{Goettingen}
\usetheme{Szeged}
\usetheme{Warsaw}

\usecolortheme{crane}

\usefonttheme{serif}
\usefonttheme{structuresmallcapsserif}

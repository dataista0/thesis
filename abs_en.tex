%\begin{center}
%\large \bf \runtitle
%\end{center}
%\vspace{1cm}
\chapter*{\runtitle}
Question answering is a computer science area that aims to generate concrete responses to questions posed in some natural language. It's a complex area that combines information retrieval, natural language processing and information extraction tools. For example, for the input \textit{\dq{`When was Noam Chomsky born?}}, a question answer system should return something like \dq{\textit{December 7th, 1928}}.
This area represents a logical step beyond the standard information retrieval systems and in the recent years it has achieved a serie of important milestones, driven by the general project of semantic web. Watson, the system developed by IBM which defeated the best human competitors of Jeopardy! is the most visible example, but even search engines like Bing! and Google have started to incorporate this kind of algorithmics.

In this thesis we research the different problems subsumed under the concept of question answering and we reviews different solutions and models applied to resolve them, under the project of the implementation of two basic systems of question answering. The first implemented system is a closed (specific) domain model with structured data only for English. The second model is a open domain multilingual system which uses as corpora wikipedias in different languages. For the first model we oriented our develpmnet following the theoretical framework exposed in the paper \cite{QADB1} and we implemented solutions for a restricted set of questions. For the second model, we used a subset of problems of the competition CLEF '07 and we developed the system using as baseline the framework Qanus, adapting it to use the multilingual natural language processing tools of the library Freeling.
\bigskip

\noindent\textbf{Keywords:} Question Answering, Closed Domain, Open Domain, Multilingual, Freeling, Qanus, CLEF, Semantic Tractability
%\begin{center}
%\large \bf \runtitulo
%\end{center}
%\vspace{1cm}
\chapter*{\runtitulo}

Question Answering es un área de ciencias de la computación que combina herramientas de búsqueda y recuperación
de la información (\textit{information retrieval}) y de procesamiento del lenguaje natural (\textit{nlp}), buscando
generar respuestas a preguntas formuladas en algún lenguaje humano.
Por ejemplo, para el input \textit{\dq{?`Cuándo nació Noam Chomsky?}} un sistema de QA debe devolver \dq{\textit{7 de diciembre de 1928}}. 
Existen varios sistemas de QA conocidos: quizás los más populares sean Wolfram Alpha e IBM-Watson, el sistema que venció a los campeones humanos del 
programa de televisión estadounidense Jeopardy! en tiempo real. Desde hace un tiempo, el buscador de Google está implementado también, lentamente
procesos de QA. El campo de investigación está muy lejos de cerrarse y los resultados son aún muy básicos debido a la complejidad inherente al problema. 

En esta tesis investigamos diferentes enfoques al problema de question answering y diseñamos e implementanos un modelo simple de question answering closed domain para una base de datos y otro open domain sobre wikipedia, ambos bilingües. 
\bigskip

\noindent\textbf{Palabras claves:} Question Answering, Closed Domain, Open Domain, Multilenguaje, Information Retrieval, Question processing, Procesamiento del lenguaje natural

%\begin{center}
%\large \bf \runtitulo
%\end{center}
%\vspace{1cm}
\chapter*{\runtitulo}
Question answering es un área de ciencias de la computación que busca generar respuestas concretas a preguntas expresadas en algún lenguaje natural. Es un área compleja que combina herramientas de búsqueda y recuperación de la información (\textit{information retrieval}), de procesamiento del lenguaje natural (\textit{nlp}) y de extracción de información (\textit{information extraction}). Por poner un ejemplo: para el input \textit{\dq{?`Cuándo nació Noam Chomsky?}} un sistema de question answering debería devolver algo como \dq{\textit{el 7 de diciembre de 1928}}.
Este área representa el paso lógico posterior a los sistemas de recuperación de documentos y logró en los último años una serie de hitos impulsados por el proyecto general de la web semántica. Watson, el sistema desarrollado por IBM que derrotó a los mejores competidores de Jeropardy! es el ejemplo más visible, pero incluso buscadores como Bing! y Google comienzan a incorporar este tipo de algoritmia.

En esta tesis investigamos los distintos problemas que se subsumen bajo el concepto de question answering y reseñamos diferentes soluciones y modelos aplicados para resolverlos, bajo el proyecto de la implementación de dos sistemas básicos de question answering. El primer sistema implementado es un modelo de dominio cerrado (específico) y datos estructurados solo para inglés. El segundo modelo es un sistema multilingüe, de dominio abierto y que utiliza como corpora las wikipedias de diferentes idiomas. Para el primer modelo orientamos nuestro desarrollo de acuerdo al modelo teórico del paper \cite{QADB1} e implementamos soluciones para un conjunto restringido de preguntas.  Para el segundo modelo utilizamos  un subconjunto de los problemas de la competencia CLEF '07 y desarrollamos el sistema utilizando como baseline el framework Qanus, adaptándolo para utilizar herramientas de procesamiento de lenguaje multilingües de la librería Freeling.
\bigskip

\noindent\textbf{Palabras claves:} Question Answering, Closed Domain, Open Domain, Multilenguaje, Information Retrieval, Question processing, Procesamiento del lenguaje natural, Freeling, Qanus, CLEF
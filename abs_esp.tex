%\begin{center}
%\large \bf \runtitulo
%\end{center}
%\vspace{1cm}
\chapter*{\runtitulo}
\horrible
Question answering es un área de ciencias de la computación que busca generar respuestas concretas a preguntas expresadas en algún lenguaje natural. Es un área compleja que combina herramientas de búsqueda y recuperación de la información (\textit{information retrieval}), de procesamiento del lenguaje natural (\textit{nlp}) y de extracción de información (\textit{information extraction}). Por poner un ejemplo: para el input \textit{\dq{?`Cuándo nació Noam Chomsky?}} un sistema de question answering debería devolver \dq{\textit{7 de diciembre de 1928}}.

Existen varios sistemas de QA conocidos: quizás los más populares sean Wolfram Alpha e IBM-Watson, el sistema que venció a los campeones humanos del 
programa de televisión estadounidense Jeopardy! en tiempo real. Desde hace un tiempo, el buscador de Google está implementado también, lentamente
procesos de QA. El campo de investigación está muy lejos de cerrarse y los resultados son aún muy básicos debido a la complejidad inherente al problema. 

En esta tesis investigamos diferentes soluciones y modelos de question answering, bajo el proyecto de la implementación de dos sistemas multilingües, uno closed domain estructurado y otro open domain, utilizando las wikipedias como corpus de cada idioma. 
\bigskip

\noindent\textbf{Palabras claves:} Question Answering, Closed Domain, Open Domain, Multilenguaje, Information Retrieval, Question processing, Procesamiento del lenguaje natural

\chapter{Agradecimientos}
A mis viejos y a mis hermanos; a las becas TICS; a mi director; a mis compañeros. - ¡7 años!.

\chapter{Introducción}
La tesis se articula así: En la Introducción explico como se articula la tesis, introduciendo el concepto general de lo que significa el question answering, con un cierto grado de detalle (categorías de datos y corpus), enfasis en multilenguaje. 
El QA está parado sobre dos ramas más amplias: Information retrieval y procesamiento de lenguajes. En el capitulo II - Marco Teórico, se introducen los conceptos básicos de estas dos ramas. Acá cuento muy a vuelo de pajaro de que se habla ahí.
En el capitulo III, "Estado de Arte", paso revista del estado de arte del área, comentando el nucleamiento de la investigación en torno a las competencias TREC y CLEF, también señalando a IBM y comentando el devenir de las evaluaciones. En "Literatura científica y sistemas" hago una reseña explicita del "Survey". Además, menciono a OpenEphyra, Just.Ask y el estado deplorable del entorno. También hago una reseña explícita del paper de Interfaz a DB y "Structured Data in IBM-Watson". 
En el capitulo IV, "Implementación de datos estructurados" comento el inicio del proyecto, la motivación de este tipo de sistemas y presento un modelo final, simple, pero elegante y funcionado. Hachando todo lo que se deba y pueda hachar. 
En el capitulo V, "Implementación sobre QANUS" viene toda la papota. Presento los ejercicios seleccionados de la CLEF 07 y hago un análisis de las wikipedias y de las preguntas. Explico qué me quedo y qué saco. Explico la adaptación a Freeling y lo copado de que sea multilingüe. Presento los datos de una manera más o menos vistosa.
En el capitulo VI, "Conclusiones y trabajo futuro" comentos las virtudes y las limitaciones del experimento y hablo de estado del campo con más descriptividad que optimismo. Sobre trabajo futuro puedo sugerir muchas mejoras técnicas. 
En el Apéndice presento librerías y datos de implementación muy burdos
En la bibliografía van los textos que leí.
\chapter{Marco teórico}
\section{General}
\subsection{Tokens}
\subsection{Feature, Tag}
\subsection{N-gramas}
\section{Information Retrieval}
\subsection{Indices invertidos}
\section{Natural Language Processing}
\subsection{POS-Tagging}
\subsection{NERC}
\subsection{Question Classification}
\subsection{Other common nlp tasks}
\chapter{Estado de arte}
\section{Competencias del área}
\section{Literatura científica y sistemas}
\chapter{Implementación de datos estructurados}
\section{QA como interfaz a una DB}
\subsection{Integración en sistemas open domain}
\section{Sistema}
\subsection{Dominio de datos}
\subsection{Modelo}
\chapter{Implementación sobre QANUS}
\section{Ejercicio de CLEF}
\subsection{Tareas}
\subsection{Corpus}
\subsection{Preguntas}
\section{Sistema}

\subsection{Information Base Processing}
\subsection{Question Processing}
\subsection{Information Retrieval}
\subsubsection{Query Generation}
\subsubsection{Passage Extraction}
\subsubsection{Lasso Heuristics}
\subsection{Answer Retrieval}
\subsubsection{Qanus baseline}
\subsubsection{Heuristic 1}
\subsubsection{Heuristic 2}
\subsubsection{Heuristic 3}
\section{Evaluación General}
\chapter{Conclusiones y trabajo futuro}

%\label{subsec:mitic}

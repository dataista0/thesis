\chapter{Estado de arte}
\section{Competencias del área}
\section{Literatura científica y sistemas}


% \chapter{Estado de Arte}
% \label{chap:estado-de-arte}
% En este capítulo analizaremos diferentes investigaciones sobre question answering para poder ilustrarnos acerca de los distintos problemas y las distintas formas de encararlos que existen. Discutiremos {\color{red}primero} una serie de investigaciones académicas pequeñas, presentadas en general en congresos y competencias del área, para definir un modelo más o menos estándar del dominio de problemas y los acercamientos típicos. Luego, comentaremos el sistema de IBM -Watson- para ver, más allá de los enfoques usuales, un enfoque exitoso. Finalmente pasaremos revista de una serie de sistemas disponibles para facilitar la creación de modelos de question answering, haciendo foco en Qanus (un framework que de hecho pudimos utilizar durante un periodo de nuestra investigación) y mencionando algunos sistemas que evaluamos teóricamente pero que, lamentablemente, no estaban disponibles \textit{out of the box}, principalmente debido a restricciones y modificaciones en el acceso programático que implementaron los buscadores populares en los últimos años\footnote{Ver por ejemplo, Google Search API, deprecado el 1ero de noviembre de 2010 aquí: https://developers.google.com/web-search/}. 


% \section{Dominio de problemas}
% QA, NLP, IR, Open domain y closed domain. Historia: primeros enfoques, BASEBALL y LUNAR. 
% En esta sección comentaremos algunos enfoques conocidos para resolver los problemas de question answering open domain y closed domain. 

% \subsection{Enfoques sobre open domain}
% La investigación, la producción de software y la formación de una comunidad de i+d vinculada con el question answering estuvo impulsada fuertemente por la competencia TREC. Esta competencia duró de tal año a tal año, momento en el que se suspendió. CLEF. NCTIC. 
% Los diferentes autores coiniciden en un modelo general de tres módulos principales: 
% \begin{itemize}
% \item Interfaz a una base de conocimientos
% \item Procesamiento lingüístico de la pregunta
% \item Búsqueda, extracción y retorno de la respuesta
% \end{itemize}
% El primer módulo consiste, por lo general, en algún soporte de documentos de information retrieval, como por ejemplo, un indice Lucene o IRQI[x]. 
% El segundo piripipí. El tercero piripipí. Por su parte, cada módulo define una serie de problemas y enfoques distintos. Subproblemas de cada problema y enfoques típicos. 

% \subsubsection{Métricas}


% \subsection{QA como interfaz a una base de datos}



% \section{IBM-Watson}

% Watson\cite{WATSON1}\cite{WATSON2} es un sistema diseñado por IBM con el objetivo de competir en
% tiempo real en el programa de televisión estadounidense Jeopardy,
% logrando resultados del nivel de los campeones humanos de este
% programa.

% El proyecto demoró 3 años de investigación, en los cuales se
% logró obtener la performance esperada (nivel humano experto) en
% cuanto a precisión, confiabilidad y velocidad, logrando derrotar a
% dos de los hombre con mayores récords históricos del show en un
% programa en vivo\footnote{En esta url está disponible el programa en el que el sistema vence a sus competidores humanos: \url{http://www.youtube.com/watch?v=WFR3lOm_xhE}} en febrero de 2011.

% El objetivo del \ proyecto puede considerarse una extensión de lo que
% fue Deep Blue, el sistema que logró el nivel de los expertos humanos
% en el ajedrez, porque buscó superar un reto que significativo y
% visible del campo de la Inteligencia Artificial tanto para la comunidad
% científica como para la sociedad en general:
% {\textquotedblleft}?`puede un sistema computacional ser diseñado para
% competir con los mejores hombre en alguna tarea que requiera altos
% niveles de inteligencia humana y, si es el caso, que clase de
% tecnología, algoritmos e ingenieria se
% requiere?{\textquotedblright}\footnote{Traducción propia de
% un fragmento de \cite{WATSON1}, p. 2}

% Watson es la implementación específica para participar en este
% programa de una arquitectura más genérica de question answering,
% DeepQA, que da el nombre al proyecto de la corporación. Esta
% arquitectura es de construcción reciente y ejemplifica perfectamente la complejidad del problema de
% QA de dominio abierto e incorpora tecnologías de punta de distintos
% dominios de ciencias de la computación, y de IA en particular:
% information retrieval, natural language processing, knowledge
% representation and reasoning, machine learning e interfaces humano -
% computadora. En el transcurso de esta tesis, IBM lanzó el programa
% \sq{Watson Ecosystem} (en noviembre de 2013) que promete la utilización
% de tecnología de punta para aplicaciones creadas por la comunidad\footnote{
% Announcing the IBM Watson Ecosystem Program: \url{http://www-03.ibm.com/innovation/us/watson/}}.

% \subsection{El problema}

% Watson debe realizar tareas como parsing, question classification,
% question descomposition, automatic source adquisition and evaluation,
% entity and relation detection, logical form generation, knowledge
% representation and reasoning manteniendo ciertos atributos de calidad
% bastante exigentes derivados de la naturaleza del show. Estas
% restricciones son:

% \begin{itemize}
% \item Confiabilidad de la respuesta: \newline
% Jeopardy tiene tres participantes con un pulsador y el que desee
% responder debe pulsar antes que los demás. Además, existe una
% penalización por respuestas incorrectas, por lo que es esencial que
% el sistema pueda determinar la confiabilidad de la respuesta obtenida a
% fin de optar por responder o no responder.
% \item Tiempos de respuesta: \newline
% La confiabilidad de la respuesta, o al menos una estimación, debe
% calcularse antes de que pase el tiempo para decidir responder (6
% segundos) y también de que otro participante oprima su pulsador
% (menos de 3 segundos).
% \item Precisión:\newline
% El tipo de respuestas que se dan en el show suelen ser respuestas
% exactas (por ejemplo: solamente un nombre, un número o una fecha,
% etc). 
% \end{itemize}

% \bigskip

% El sistema cuenta con varios componentes heurísticos que estiman
% ciertos features y grados de confiabilidad para diferentes respuestas,
% los cuales son evaluados por un sistema general que sintetiza un grado
% de confiabilidad para una respuesta final y determina así si
% responder o no responder. 

% El programa consta de un tablero con 30 pistas (o preguntas) organizadas
% en seis columnas, cada una de las cuales es una categoría. Las
% categorías van desde temas acotados como
% {\textquotedblleft}historia{\textquotedblright} o
% {\textquotedblleft}ciencias{\textquotedblright} hasta temas más
% amplios como {\textquotedblleft}cualquier cosa{\textquotedblright} o
% {\textquotedblleft}potpourri{\textquotedblright}. Watson intenta
% respuestas sobre varias hipótesis de dominio y verifica en cual de
% ellos se logran respuestas de mayor confiabilidad. 

% Por otra parte, el grueso de las preguntas de Jeopardy son del tipo
% \textit{factoid}, esto es, preguntas cuya respuesta esta basada en
% información fáctica acerca de una o más entidades individuales.


% \bigskip

% Por ejemplo:

% Categoría: Ciencia General

% Pista: Cuando es impactado por electrones, un fósforo emite energía
% electromagnética de esta forma

% Respuesta: Luz (o fotones)


% \bigskip

% A su vez, existen ciertos tipos de pistas que requieren un enfoque
% particular, por ejemplo, pistos que constan de dos subpistas muy
% débilmente relacionadas, o problemas matemáticos formulados en
% lenguaje humano, o problemas de fonética, etc, que no pueden ser
% simplemente dejados de lado porque, si bien tiene poca probabilidad de
% aparición, cuando aparecen lo hacen en bloque y pueden arruinar el
% juego de Watson. Se acordó con la productora del programa, sin
% embargo, dejar de lado preguntas audiovisuales (aquellas que presentan
% una imagen o un audio y requieren interpretarlo) y preguntas que
% requieren instrucciones verbales del presentador.


% \bigskip

% Para determinar el dominio de conocimiento, los investigadores
% analizaron 20000 preguntas, extrayendo su LAT (lexical answer type, o
% tipo léxico de respuesta). El LAT se define como una palabra en la
% pista que indica el tipo de la respuesta esperado. Por ejemplo, para la
% pista {\textquotedblleft}Investanda en 1500{\textquoteright}s para
% agilizar el juego, este movimiento involucra dos
% piezas{\textquotedblright} el LAT es
% {\textquotedblleft}movimiento{\textquotedblright}. Menos del 12\% de
% las pistas no indicaba explícitamente ningún LAT, usando palabras
% como {\textquotedblleft}esto{\textquotedblright} o
% {\textquotedblleft}eso{\textquotedblright}. En estos casos, el sistema
% debe inferir el tipo de respuesta del contexto. Del análisis de estas
% 20000 pistas se reconocieron 2500 tipos léxicos distintos, de los
% cuales los 200 más frecuentes no llegaban a cubrir el 50\% del total
% de pistas. Esto implica que un approach estructurado (orientado por el
% tipo de respuesta), si bien resulta útil para algunos tipos, no es
% suficiente para abordar el problema completo.

% \subsection{Métricas}

% Las métricas de resultados, además del tiempo de respuesta, son la
% \textit{precisión} (preguntas contestadas correctamente / preguntas
% contestadas) y el \textit{porcentaje de respuestas dadas }(preguntas
% contestadas / total de preguntas). Mediante la configuración de un
% threshold de \textit{confiabilidad} pueden obtenerse distintas
% estrategias de juego: un umbral bajo repercutirá en un juego más
% agresivo, incrementando la proporción de respuestas contestadas,,
% pero disminuyendo su precisión, mientras que un umbral alto
% determinará un juego conservador, con menos respuestas dadas pero
% mayor precisión en las mismas. Es un clásico escenario de trade-off
% entre dos atributos de calidad. Un buen sistema de estimación de
% confiabilidad implica una mejora general del sistema, aún cuando el
% módulo de generación de respuestas permanezca idéntico.


% \bigskip

% En el show, el porcentaje de respuestas dadas depende de la velocidad
% con la que se llega a presionar el pulsador, lo cual sólo interesa
% para el dominio de QA como una restricción temporal. 


% \bigskip

% Mediante análisis numérico, los investigadores determinaron que los
% campeones de Jeopardy lograban tomar entre el 40\% y el 50\% de las
% preguntas y, sobre ellas, lograban una precisión de entre el 85\% y
% el 95\%, lo que determinaba una barrera de performance bastante
% exigente en lo que respecta a QA.


% \bigskip

% \subsection{Baseline}

% El equipo de IBM intentó utilizar dos sistemas consolidados en QA y
% adaptarlos al problema \ de Jeopardy. \ El primero fue PIQUANT
% (Practical Intelligent Question Answering Technology), un sistema
% desarrollado por IBM en conjunto con el programa del gobierno
% estadounidense AQUAINT y varias universidades, que estaba entre los
% mejores según la TREC (Text Retrieval Conference), una autoridad en
% el área. PIQUANT consta de un pipeline típico (véase QANUS) con
% tecnología de punta, logrando un rango del 33\% de respuestas
% correctas en las evaluaciones TREC-QA. Los requerimientos de la
% evaluación de TREC son muy distintos de los de Jeopardy: TREC ofrece
% un corpus de conocimiento relativamente pequeño (1M de documentos) de
% donde las respuestas deben ser extraídas y justificadas, el tipo de
% preguntas de TREC son menos complejas a nivel ling\"uístico que las
% de Jeopardy y la estimación de confiabilidad no resulta una métrica
% importante (dado que no hay penalización por respuestas incorrectas).
% Además, los sistemas tienen permitido acceder a la web y las
% restricciones temporales son, por mucho, más amplias (por ejemplo:
% una semana para responder 500 preguntas). En Jeopardy, además de las
% restricciones ya mencionadas, un requerimiento fue que el sistema
% trabaje sobre datos locales y no acceda a la web en tiempo real. El
% intento de adaptar PIQUANT al problema de Jeopardy dio pésimos en
% comparación con los necesarios: 47\% de precisión sobre el 5\% de
% respuestas con mayor confiabilidad y 13\% de precisión en general. 

% Por otro lado, el equipo intentó adaptar el sistema OpenEphyra
% (véase OpenEphyra), un framework open-source de QA desarrollado en
% CMU (Carnegie Mellon University) basado en Ephyra (no libre),
% diseñado también para la evaluación TREC. OpenEphyra logra un
% 45\% de respuestas correctas sobre el set de datos de evaluación TREC
% 2002, usando busqueda web. La adaptación resultó aún peor que la
% de PIQUANT (con menos del 15\% de respuestas correctas y una mala
% estimación de la confiabilidad). 

% Se probaron dos adaptaciones de estos sistemas. una basada en
% búsquedas de texto puro y otra basada en reconocimiento de entidades.
% En la primera, la base de conocimiento se modeló de manera no
% estructurada y las preguntas se interpretaron como términos de una
% query, mientras que en la segunda se modeló una base de conocimientos
% estructurada y las preguntas se analizaron semánticamente para
% reconocer entidades y relaciones, para luego buscarlos en la base.
% Comparando ambos enfoques en base al porcentaje de respuestas dadas, el
% primero dio mejores resultados para el 100\% de las respuestas,
% mientras que la confiabilidad general era baja; por otro lado, el
% segundo enfoque logró altos valores de confiabilidad, pero sólo en
% los casos en que efectivamente logra identificar entidades. De aquí
% se infiere que cada enfoque tiene sus ventajas, en el dominio de
% problemas apropiado.

% \subsection{La arquitectura DeepQA}
% \label{subsec:deep-qa}
% Los intentos de adaptación iniciales, como vimos, no dieron
% resultados, así como tampoco sirvieron las adaptaciones de algoritmos
% de la literatura científica, los cuales son realmente difíciles de
% sacar de su contexto original y de las evaluaciones sobre las cuales
% fueron testeados. Este problema, veremos -por ejemplo, con QANUS y
% Reverb- , se repitió en nuestro proyecto. Como conclusión de estos
% intentos frustrados, el equipo de IBM entendió que una arquitectura
% de QA no debía basarse en sus componentes concretos sino en la
% facilidad para incorporar nuevos componentes y para adaptarse a nuevos
% contextos. Así surgió DeepQA, la arquitectura de base, de la cual
% Watson es una instancia concreta para un contexto particular (con
% requerimientos de alta precisión, buena estimación de
% confiabilidad, lenguaje complejo, amplitud de dominio y restricciones
% de velocidad). DeepQA es una arquitectura de computo paralelo,
% probabilistico, basado en recopilación de evidencia y scoring. Para
% Jeopardy se utilizaron más de 100 técnicas diferentes para analizar
% lenguaje natural, identificar y adjudicar valor a fuentes de
% información, encontrar y generar hipótesis, encontrar y rankear
% evidencias y mergear y rankear hipótesis en función de esta
% evidencia. La arquitectura sirvió para ganar Jeopardy, pero también
% se adaptó a otros contextos como la evaluación TREC, dando
% resultados mucho mejores que sus predecesores. Los principios de
% diseño subyacentes de la arquitectura son:

% \begin{itemize}
% \item Paralelismo masivo\newline
% Para evaluar distintas hipótesis en distintos dominios con poco
% acoplamiento.
% \item Pervasive confidence estimation:\newline
% Ningún componente genera la respuesta final, sino que da una serie de
% features y grados de confiabilidad y evidencia para distintas
% hipótesis, que luego son sintetizados.
% \item Integrate shallow and deep knowledge:
% \end{itemize}

% \bigskip

% A continuación, enumeraremos la lista de pasos que sigue el sistema
% para obtener la respuesta a una pregunta:

% \subsubsection{Adquisición de contenidos}

% El primer paso de DeepQA es la adquisición de contenidos. Este paso es
% el único que no se realiza en tiempo de ejecución y consiste en
% crear la base de conocimiento en la cual el proceso final buscará la
% respuesta a la pregunta, combinando subprocesos manuales y
% automáticos. 

% En principio se caracteriza el tipo de preguntas a responder y el
% dominio de aplicación. El análisis de tipos de preguntas es una
% tarea manual, mientras que la determinación del dominio puede
% encararse computacionalmente, por ejemplo, con la detección de LATs
% que señalamos antes. Dado el amplio dominio de conocimientos que
% requiere Jeopardy, Watson cuenta con una gran cantidad de
% enciclopedias, diccionarios, tesauros, artículos académicos y de
% literatura, etc. A partir de este corpus inicial, el sistema busca en
% la web documentos relevantes y los relaciona con los documentos ya
% presentes en el corpus. 

% Además de este corpus de documentos no estructurados, DeepQA maneja
% contenidos semi-estructurados \ y estructurados, incorporando bases de
% datos, taxonomías y ontologías como dbPedia, Wordnet y las
% ontologías de Yago. 

% \subsubsection{Análisis de la pregunta}

% El primer paso en run-time es el análisis de la pregunta. En este paso
% el sistema intenta entender qué es lo que la pregunta está
% preguntado y realizar los primeros análisis que determinan cómo
% encarará el procesamiento el resto del sistema. Watson utiliza
% shallow parses, deep parses, formas lógicas, pos-tags,
% correferencias, detección de entidades nombradas y de relaciones,
% question classification, además de ciertos análisis concretos del
% domiento del problema.

% En este proceso se clasifica el tipo de la pregunta (los tipos están
% determinados por el show: puzzles, matemáticos, etc). También se
% busca el tipo de respuesta esperada, dónde los tipos manejados son
% por Watson son los LATs extraídos de las preguntas de ejemplo. El LAT
% determina el {\textquotedblleft}tipo{\textquotedblright} de la
% respuesta, que clase de entidad \textit{es} la respuesta (una fecha, un
% hombre, una relación, etc). El equipo de IBM intentó adaptar
% distintos algoritmos de clasificación preexistentes, pero después
% de intentar entrenarlos para el dominio de tipos de Jeopardy, llegaron
% a la conclusión de que su eficacia era dependiente del su sistema de
% tipos default, y que la mejor forma de adaptación era mappear su
% output a los tipos utilizados por Watson (un enfoque similar fue
% utilizado en esta tesis con respecto al clasificador de Stanford). Otra
% anotación importante es el
% {\textquotedblleft}foco{\textquotedblright} de la pregunta, la parte de
% la pregunta tal que si se la reemplaza por la respuesta, la pregunta se
% convierte en una afirmación cerrada.

% Por ejemplo, para {\textquotedblleft}El hombre que escribió Romeo y
% Julieta{\textquotedblright}, el foco es {\textquotedblleft}El hombre
% que{\textquotedblright}. Este fragmento suele contener información
% importante sobre la respuesta y al reemplazarlo por una respuesta
% candidata se obtiene una afirmación fáctica que puede servir para
% evaluar distintos candidatos y recolectar evidencia. Por ejemplo,
% reemplazando por distintos autores y verificando que la oración
% resultante esté presente en el corpus.

% Por otro lado, muchas preguntas involucran relaciones entre entidades y,
% más puntualmente, tienen una forma sujeto-verbo-objeto. Por ejemplo,
% tomando la pista anterior, podemos extraer la relación
% \textit{escribir(x, Romeo y Julieta)}. La amplitud del dominio de
% Jeopardy hace que la cantidad de entidad y de relaciones entre
% entidades sea enorme, pero esto empeora aún más al considerar las
% distintas formas de expresar la misma relación. Por eso, Watson
% sólo logra encontrar directamente una respuesta mediante
% reconocimiento de entidades y relaciones sobre el 2\% de las pistas. En
% general, este tipo de enfoque es útil sobre dominios más acotados,
% mientras que la detección de relaciones como approach general a un
% problema de question answering de dominio amplio es un área de
% investigación abierta. 

% Una particularidad ya señalada de las preguntas de Jeopardy son las
% pistas con subpistas no relacionadas. Para atacar este problema, Watson
% genera distintas particiones y resuelve todas en paralelo, sintetizando
% las respuesta de cada partición generada mediante algoritmos ad-hoc
% de ponderación de confiabilidad y otras características.

% \subsubsection{Generación de hipótesis}

% El tercer paso (segundo en run-time) es la generación de hipótesis:
% tomando como input el resultado del paso anterior se generan respuestas
% candidatas a partir de la base de conocimiento offline. Cada respuesta
% candidata reemplazada por el foco de la pregunta es considerada una
% hipótesis, que el sistema luego verificará buscando evidencias y
% adjudicando un cierto grado de confiabilidad.

% En la búsqueda primaria de respuestas candidatas, se busca generar
% tantos pasajes como sea posible. El resultado final obtenido revela que
% el 85\% de las veces, la respuesta final se encuentra entre los
% primeros 250 pasajes devueltos por la búsqueda primaria. La
% implementación utiliza una serie variada de técnicas, que incluyen
% diferentes motores de búsqueda de textos (como Indri y Lucene),
% búsqueda de documentos y de pasajes, búsquedas en bases de
% conocimiento estructuradas como SPARQL con triple store y la
% generación de mutiples queries a partir de una sola pregunta. La
% búsqueda estructurada de triple stores depende del reconocimiento de
% entidades y relaciones del paso anterior.

% Para un número pequeño de LATs, se definió una suerte de conjunto
% de entidades fijas (por ejemplo: países, presidentes, etc). Si la
% respuesta final no es retornada en este paso, entonces no hay
% posibilidad de obtenerla en los siguiente. Por eso se prioriza el
% recall sobre la precisión, con el supuesto de que el resto del
% pipeline logrará filtrar la respuesta correcta correctamente. Watson
% genera varios cientos de hipótesis candidatas en este paso.


% \bigskip

% \subsubsection*{(Soft filtering)}

% Para optimizar recursos, se realiza un filtrado liviano de respuestas
% antes de pasar a la recopilación de evidencia y al scoring de
% hipótesis. Un filtrado liviano es, por ejemplo, comprobar similaridad
% de la respuesta candidata con el LAT esperado de la respuesta. Aquellas
% hipótesis que pasan el filtro pasan al siguiente proceso, que realiza
% un análisis más exhaustivo.


% \bigskip

% \subsubsection{Recuperación de evidencias y scoring de pasajes}

% Para recuperar evidencias se utilizan varios algoritmos. Uno
% particularmente útil es buscar la hipótesis candidata junto con las
% queries generadas por la pregunta original, lo que señala el uso de
% la respuesta en el contexto de la pregunta. \ Las hipótesis con sus
% evidencias pasan al siguiente paso, dónde se les adjudica un score. 

% El proceso de scoring es donde se realiza la mayor parte del análisis
% más fuerte a nivel computacional. DeepQA permite la incorporación
% de distintos Scorers, que consideran diferentes dimensiones en las
% cuales la hipótesis sirve como respuesta a la pregunta original. Esto
% se llevó a cabo definiendo una interfaz común para los scorers.
% Watson incorpora más de 50 componentes que producen valores y
% diferentes features basados en las evidencias, para los distintos tipos
% de datos disponibles (no estructurados, semi estructurados y
% estructurados). Los scorers toman en cuenta cuestiones como el grado de
% similaridad entre la estrurctura de la respuesta y de la pregunta,
% relaciones geoespaciales y temporales, clasificación taxonómica,
% roles léxicos y semánticos que se sabe que el candidato puede
% cumplir, correlaciones entre terminos con la pregunta, popularidad (u
% obscuridad) de la fuente del pasaje, aliases, etc.

% POR EJEMPLO: COPIAR NIXON

% Los distintos scores se combinan luego en un score único para cada
% dimensión.

% (Merge)

% Recién después de este momento, Watson realiza un merge entre
% hipótesis idénticas. Las hipótesis idénticas son diferentes
% formulaciones ling\"uisticas de lo mismo, por ejemplo:
% {\textquotedblleft}X nació en 1928{\textquotedblright} o
% {\textquotedblleft}El año de nacimiento de X es
% 1928{\textquotedblright}. Finalmente, se procede a estimar un ranking
% único y una confiabilidad única para las distintas hipótesis. En
% este paso se utilizan técnicas de machine learning que requieren
% entrenamiento, y modelos basados en scores. Se utilizan técnicas
% jerárquicas como mixture of experts y stacked generalization y,
% finalmente, un metalearner fue entrenado para ensamblar los distintos
% resultados intermedios. 


% \bigskip

% \subsection{Tiempos y escala}

% DeepQA utiliza Apache UIMA, un framework que implementa UIMA
% (Unestructured Information Management Architecture): todos los
% componentes de DeepQA son IUMA-annotators, módulos que producen
% anotaciones y aserciones sobre un texto.

% La implementación inicial de Watson corría sobre un sólo
% procesador y demoraba aproximadamente 2 horas en contestar una sola
% pregunta. La arquitectura paralela permite, sin embargo, que al
% correrlo sobre 2500 núcleos -de la implementación final- los
% tiempos de respuesta oscilen entre 3 y 5 segundos, que es lo esperado.

% Finalmente, la implementación de Watson logró alcanzar el estándar
% de resultados de los campeones de Jeopardy y, como ya dijimos,
% compitió y ganó el programa en Febrero de 2011. Además, se
% realizaron adaptaciones para trabajar sobre los problemas de TREC, en
% los cuales se demostró una amplia mejoría en comparación con
% PIQUANT y OpenEphyra


% \bigskip

% \subsection{Conclusiones sobre IBM-Watson}

% System level approach.


% \bigskip


% \bigskip

% \section{Otros sistemas de QA: OpenEphyra, Aranea y Just.Ask}

% \bigskip

% El paper describe las arquitecturas de todos los sistemas, si sirve
% meter más info

% El paper [EPHYRA1] \ busca crear un criterio cuantitativo para comparar
% la eficacia de distintos pasos de Just.Ask, Open Ephyra y Aranea
% basándose en la arquitectura \textit{pipeline de tres pasos}
% compartida por todos. Los tres sistemas, por lo demás, están
% basados en la web, utilizando distintas APIs de buscadores o bien
% analizando los resultados de la interfaz de usuario de los mismos. 

% El primer ítem importante a destacar de este trabajo, es que, al
% momento de la experimentación \textbf{Aranea no funcionaba más y
% estaba discontinuado}\footnote{\ (Resaltado en Sección 8, muy
% concluyente).\par }\textbf{. }El autor se comunicó con el responsable
% del proyecto que corroboró que las APIs de los buscadores en los que
% se basaba Aranea cambiaron y no había interés en readaptar el
% código para que vuelva a funcionar. Las comparaciones que logró
% entre Just.Ask y Open Ephyra son interesantes y concluyentes a favor de
% la performance de OpenEphyra. 

% (Freeling \ + Cambio de Base + Rigidez)

% \section{Algunos ejemplos académicos}

% \begin{itemize}
% \item Watson: \cite{WATSON1} y \cite{WATSON2}
% \item Qanus: \cite{QANUS1}

% \item Ephyra: \cite{EPHYRA1}
% \item Varios de QA: Yago \cite{YAGO-QA1}, sobre una teoria de QA como interfaz a DBs: \cite{QADB1}. Corpus: \cite{TRAIN-QA1}, qall-me: \cite{QALL-ME1}, practical QA: \cite{QAS1}, simple QA: \cite{QAS2} y Surface de Ravishandran: \cite{SURF1}. Introducción a QA: \cite{QA1} y \cite{QA2} y \cite{QA3}
% \item Aranea: \cite{ARANEA1} (no leido)
% \item Passage retrieval evaluation: \cite{PASSAGE1}
% \item Evaluacion de las TREC8 (metrica de \cite{QA3} LASSO): \cite{TREC8}
% \end{itemize}

% \section{Conclusiones}

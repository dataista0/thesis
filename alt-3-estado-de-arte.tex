\chapter{Estado de Arte}
\label{chap:estado-de-arte}
\section{Introducción general}
\label{sec:intro-general-qa}
\cerrada
Como vimos brevemente en la introducción, QA es un área de investigación de ciencias de la computación que busca generar respuesta concretas a preguntas expresadas en algún lenguaje natural. Es, por esto mismo, un problema complejo que involucra herramientas y modelos de otras áreas de existencia autónoma, como information retrieval (IR), procesamiento del lenguaje natural (PLN, NLP) e information extraction (IE). 

También señalamos algunos ejes de subclasificación de problemas de QA. Con respecto al dominio de hechos y conocimientos sobre los que se espera que el sistema sepa responder los sistemas se clasifican como de dominio abierto (\textit{open domain}) o de dominio cerrado (\textit{closed domain}): mientras de los primeros se espera que sepan responder pregunta acerca cualquier tema o dominio, de los segundos solo se espera que sepan responder preguntas acerca de un dominio acotado particular. A su vez, los datos pueden ser estructurados, semi estructurados o no estructurados. El ejemplo típico de una base de conocimientos estructurada es una base de datos relacional, un tipo de datos semi estructurados puede ser un documento xml no normalizado, mientras que el tipo de datos no estructurado por antonomasia es un corpus de documentos en texto plano. 

Las bases de conocimiento de los sistemas de QA pueden tener uno o más tipos de datos, pero cada unos de estos tipos determina un enfoque algorítmico diferente. Por ejemplo, si la base de conocimientos es una DB relacional con un lengueja formal de consultas similar al SQL, el problema de QA típico consiste en encontrar un mappeo desde la pregunta en lenguaje natural a una consulta formal expresada en un lenguaje comprensible para la base de datos. Esto implica que el foco de trabajo está en la generación de esta consulta y no en el trabajo posterior sobre los resultados de ejecutarla. Por otro lado, un corpus no estructurado no permite una consulta formal, por lo que el enfoque usual es obtener una lista de documentos relevantes para luego aplicar distintas técnicas de procesamiento de textos para extraer la respuesta buscada. 

Estos dos ejes de clasificación (grado de especificidad del dominio y grado de estructuración de los datos) suelen tener una correlación, a saber: los dominios cerrados suelen disponer (o permitir la construcción sencilla de) una base de conocimientos estructurada, mientras que los dominios abiertos suelen forzar datos no estructurados. Esta correlación no es una asociación. En realidad, muchos sistemas open domain actuales son hídridos para estas clasficaciones, ya que suelen combinar varias bases de conocimiento de distintos tipos. Un sistema híbrido podría tener un corpora no estructurado basada en la web para preguntas generales y, a su vez, varias bases estructuradas para dominios especificos, que permiten responder solo algunas preguntas con un resultado mejor. Otro uso puede ser extraer respuestas mediante métodos de pln sobre corpora no estructurados para luego verificar su tipo contra bases de conocimiento estructuradas generales como Freebase o dbPedia. 

En cuanto a aplicaciones comerciales masivas, podemos mencionar la aplicación Siri, el asistente personal de iOS y también la incorporación de funcionalidades de question aswering en los grandes buscadores\footnote{Buscar, por ejemplo: \dq{¿Cuándo nació Bill Gates?} en Google}.  Otros proyectos relativamente actuales que podríamos mencionar son Wolfram|Alpha\footnote{\url{www.wolframalpha.com}}, un sistema online de question answering lanzado por la compañía Wolfram Research y también a IBM-Watson, el sistema de IBM que venció a los favoritos del show de trivia estadounidense Jeopardy!

La organización de este capitulo es como sigue: en lo que queda de esta sección (\allref{sec:intro-general-qa}), repasaremos la historia de la disciplina, desde sus origenes hasta el estado actual de las competencias en torno a las cuales está hoy estructurada la investigación (\ref{subsec:historia}), luego reseñaremos las métricas utilizadas por estas competencias a la hora de evaluar la performance de los sistemas (\ref{subsec:metricas}). En la siguiente sección (\allref{sec:literatura}) pasaremos revista de diferentes investigaciones actuales sobre question answering para definir un modelo más o menos estándar del dominio de problemas y los acercamientos típicos, discutiendo diferentes investigaciones sobre QA open-domain (\ref{subsec:open-domain}), un enfoque estructurado para dominios cerrados (\ref{subsec:closed-domain}) y, finalmente el caso concreto de la implementación de IBM-Watson (\ref{subsec:ibm-watson}).


\subsection{Historia y Competencias}
\label{subsec:historia}
\label{subsec:competencias}
\cerrada
Los primeros sistemas que podemos considerar como pertenecientes al área datan de los años sesenta. Las primeras décadas de investigación al respecto están centradas en sistemas de dominio cerrado, es decir, su objetivo consiste en responder preguntas sobre temas específicos, apoyandose en general en una base de conocimientos estructurada, especificamente creada para el sistema. Por ejemplo Baseball \cite{BASEBALL}, un proyecto del MIT que data de 1961, es la primera investigación académica cuyo enfoque la incorpora con certeza en este área que pudimos encontrar en Internet. Baseball respondía preguntas sobre algunos hechos vinculados al torneo de la Liga Americana de baseball de un año (no se especifica cuál). La información se archivó de manera estructurada y contenía: el día, el mes, el lugar, los equipos y los puntajes de cada juego por un año\footnote{Como una nota de color, trascribimos las dos primeras oraciones del abstract: \dq{Baseball is a computer program that answers questions phrased in ordinary English about stored data. The program reads the question from punched cards}. Traducido: Baseball es un programa de computadora que responde preguntas verbalizadas en inglés común sobre datos guardados. El programa lee la pregunta de tarjetas perforadas.}. Otro ejemplo conocido es el chatbot-psicóloga Eliza, cuyo código original fue escrito en 1966, que se basa en reglas muy simples de matcheo de las lineas ingresadas por el usuario para simular una conversación de terapia bastante convincente\footnote{En internet están disponibles diferentes implementaciones de Eliza. En \url{https://github.com/julian3833/eliza} puede encontrarse una implementación funcional en python}. Existieron otros desarrollos e investigaciones durante las siguientes décadas, con complejidad creciente, pero con un enfoque similar al de Baseball y de Eliza. Pero la investigación, la producción de software y, principalmente, la formación de una comunidad de desarrollo e investigación mínimamente articulada vinculada con el question answering estuvo impulsada fuertemente por el lanzamiento de los ejercicios de QA en la competencia TREC\footnote{\url{http://trec.nist.gov/}} (Text Retrieval Conference) en el año 1999, la TREC-8 \cite{TREC8}.

TREC se convirtió en una referencia obligada y centro de nucleamiento de la investigación en QA. Esta primera competencia monolengua (para inglés), consistía en retornar un fragmento de texto de entre 50 y 250 bytes conteniendo la respuesta a preguntas fácticas (factoids). La evaluación, que marcó la historia futura de la evaluación de QA en competencias, consideraba el sistema de QA como un todo, es decir, no consideraba la evaluación de subcomponentes. La organización permitía dar una lista con 5 respuestas posibles por cada pregunta, y esto siguió así hasta la TREC-2002, cuando se comenzó a exigir una sola respuesta por pregunta. TREC siguió lanzando un track de QA en todas sus competencias anuales, variando la forma y la complejidad de los ejercicios propuestos, por ejemplo: incorporando preguntas largas, con un formato realista y corpora de mayor tamaño, también introduciendo preguntas sin respuesta (NIL answers) y preguntas de tipo Lista (por ejemplo: \sq{¿Qué libros escribió Friedrich Nietzsche entre 1870 y 1896?}), pregunta de definiciones (por ejemplo: \sq{¿Quién es el General Paz?}), en TREC 2003, preguntas agrupadas por un tema (target), explícito o implícito y de diferentes tipos -como personas, organizaciones, cosas, eventos, etc-, restricciones temporales (por ejemplo: antes, durante y después de una fecha, evento o período), anáforas y co-referencias entre preguntas (por ejemplo: ¿Quién es George Bush?. ¿Y quién es \textit{su} esposa?), etc. Las competencias anuales de la TREC, por otro lado cambiaron el enfoque que existía en décadas anteriores, hacia preguntas de dominio abierto y corpora de datos textuales planos. 

En lo que respecta a los sistemas multi-idioma (y mono-idioma no centrados en inglés), más allá de algunos eventos anteriores, un hito clave es el lanzamiento, en 2003, de la primer tarea de QA \cite{CLEF03} de CLEF (Cross Language Evaluation Forum)\footnote{\url{http://www.clef-initiative.eu/}} proponiendo ejercicios tanto monolingües como multilingües en varios idiomas europeos. Se propusieron ejercicios monolingües para francés, español, alemán, italiano y neerlandés, y ejercicios multilingües, con corpora en inglés y preguntas en otros idiomas europeos. En este año, se exigían respuesta de hasta 50 bytes, se permitía una lista de 3 respuestas por pregunta y los ejercicios podían incluir preguntas sin respuesta en los corpora. En la CLEF 2004, la tarea de QA principal incorporó 9 lenguas de origen (lengua de la pregunta) y 7 de destino (lengua del corpus), habilitando 55 ejercicios mono y bilingües. La cantidad de respuesta se redujo a 1 y se incorporaron preguntas de tipo "¿Cómo...?", de definiciones, de listas y también se agregaron restricciones temporales (antes, durante y después de una fecha, evento o período). CLEF continuó presentando pistas de ejercicios de QA todos los años desde entonces, con diferentes variaciones. Otra conferencia de prestigio en el área es NTCIR \footnote{\url{http://research.nii.ac.jp/ntcir/}}, que puede es un análogo a CLEF para idiomas asiáticos, incorporando tareas monoligües y bilingües entre inglés, japonés y chino, entre otras.

\subsection{Métricas de evaluación}
\label{subsec:metricas}
\cerrada
En cuanto a las métricas de evaluación utilizadas por estas competencias respecta, es importante distinguir la evaluación de las respuestas a una pregunta y la evaluación general del sistema. Con respecto a la evaluación de respuestas, la clasificación usual de evaluación consiste en 4 valores asignados a cada respuesta: Correcta (C), si responde a la pregunta y el fragmento de soporte justifica esa respuesta; No soportada (U, de Unsupported), si responde a la pregunta pero el fragmento de soporte no está o no justifica la respuesta; Inexacta (X), si sobra o falta información en la respuesta (algunas campañas solo aceptan respuestas exactas) y, finalmente, Incorrecta (I), si ninguna de las condiciones anteriores se cumple. 

La evaluación de respuestas puede ser estricta (strict) o \textit{indulgente} (lenient) dependiendo de si considera las respuestas no soportadas como válidas o como inválidas. Por otro lado, la evaluación puede ser manual -llevada a cabo por un jurado especial- o bien automática. En el caso de la evaluación automática, siempre es indulgente, ya no hay forma de decidir si un fragmento soporta o no una respuesta. Para este tipo de evaluaciones se suele utilizar un set de patrones de respuestas correctas que se deriva de un set de "respuestas conocidas" generadas por los organizadores y los mismos competidores, conocido como R-set, pero no es un proceso trivial: en principio, es dificil saber qué respuestas correctas puede esperarse, ya que por la naturaleza del problema es probable que diferentes sistemas devuelvan respuestas textualmente distintas. 

Con respecto a la métricas de evaluación para sistemas como un todo, veremos a continuación las métricas másconocidas: MRR (Mean reciprocal rank, o promedio de rankings recíprocos), CWS (Confidence Weighted Score, Puntaje ponderado por confiabilidad) y K-Measure (Medida K).

\subsubsection*{MRR - Mean Reciprocal Rank}
El MRR mide la habilidad de un sistema para responder un conjunto de pregunta $Q$. Asume que el sistema devuelve una o más respuestas por pregunta, en orden de "confiabilidad". El puntaje de cada pregunta individual ($RR(q_i)$) se define como la inversa de la posición de la primera respuesta correcta en la lista de respuestas dada, o cero si no hay ninguna.

\begin{equation}\label{eq:mrr}
 \text{MRR} = \frac{1}{|Q|} \sum_{i=1}^{|Q|} \frac{1}{RR(q_i)}. \!
\end{equation}

\begin{equation*}
    RR(q_i) = \begin{cases}
               0     & \text{si existe al menos una respuesta correcta}\\
               \frac{1}{\text{Posicion de }q_i} & \text{si }q_i\text{ es la primer respuesta correcta}\\
           \end{cases}
\end{equation*}

MRR puede tomar valores entre 0 y 1 inclusive, aunque el RR de cada pregunta tiene un espectro finito de valores (por ejemplo, para 5 respuestas puede valer 0, 0.2, 0.25, 0.33, 0.5 o 1). La primera vez que se usó MRR fue en la competencia TREC-8. En esa competencia se permitía una lista de 5 respuestas ordenadas para cada pregunta.  


\subsubsection*{CWS - Confidence Weighted Score}
CWS (de Confidence Weighted Score, Puntaje de confiabilidad ponderada, también conocido como Precisión promedio) asume que el sistema devuelve una respuesta única por pregunta pero que estas respuestas a las diferentes preguntas están ordenadas por grado de confiabilidad descendente. 
CWS recompensa respuestas correctas colocadas primero. CWS toma valores entre 0 y 1.  Uno de los principales problemas de esta métrica es que puede juzgar no la capacidad para responder preguntas sino la de ordenarlas por confiabilidad.

\begin{equation}\label{eq:cws}
 \text{CWS} = \frac{1}{|Q|} \sum_{i=1}^{|Q|} \frac{\text{Cantidad de rtas correctas en las primeras $i$ posiciones}}{i}. \!
\end{equation}

 
\subsubsection*{K-Measure}
K-Measure fue propuesta en \cite{CLEF04}. Pide a los sistemas una lista de respuestas por pregunta y un grado de confiabilidad para cada una de ellas, entre 0 y 1. 

\begin{equation}\label{eq:k}
 \text{K} = \frac{1}{|Q|} \sum_{i=1}^{|Q|} \frac{\sum_{r \in ANS_i} conf(r) \times eval(r)}{\text{max}\{|R_i|, |ANS_i|\}}. \!
\end{equation}

Para una pregunta $i$, $R(i)$ es el total de respuestas distintas conocidas, mientras que $ANS_i$ son las respuestas dadas por el sistema a evaluar para la pregunta $i$, $conf(r)$ es la confiabilidad entre 0 y 1 asignada por el sistema a la respuesta $r$ y $eval(r)$ es la evaluación hecha por la organización.


\begin{equation*}
    eval(r) = \begin{cases}
               1  & \text{si $r$ es juzgada correcta}\\
               0  & \text{si $r$ está repetida}\\
               -1 & \text{si $r$ es incorrecta}\\
           \end{cases}
\end{equation*}

Vale que $K \in \mathbb{R} \wedge  K \in [-1,1]$ , y $K=0$ se identifica con un sistema que asigna confiabilidad 0 a todas sus respuestas y, por lo tanto sirve como baseline. 

La dificultad principal de esta métrica es determinar un $R_i$ exhaustivo. Una propuesta alternativa que tiene esto en cuenta, propuesta también en \cite{CLEF04} es $K1$, que elimina la ocurrencia de $R_i$, pero solo sirve para evaluar sistemas que retornan una sola respuesta por pregunta.

\begin{equation}\label{eq:k1}
 \text{K1} = \frac{\sum_{r \in ANS_i} conf(r) \times eval(r)}{|Q|}. \!
\end{equation}

\begin{equation*}
    eval(r) = \begin{cases}
               1  & \text{si $r$ es juzgada correcta}\\
               -1 & \text{si $r$ es incorrecta}\\
           \end{cases}
\end{equation*}

Nuevamente, vale que $K1 \in \mathbb{R} \wedge  K1 \in [-1,1]$ y $K1=0$ se condice con un sistema sin conocimiento.


\section{Literatura y sistemas}
\label{sec:literatura}
\subsection{Enfoques sobre open domain}
\label{subsec:open-domain}
\falta
Existe un consenso general a la ahora de definir el modelo de software más abstracto o de arquitectura para encarar la construcción de un sistema de question answering open domain. Esta arquitectura consiste en un pipeline con al menos tres módulos o compotentes bien diferenciados:
\begin{itemize}
\item Módulo de procesamiento de la pregunta
\item Módulo de procesamiento de documentos
\item Módulo de procesamiento de la respuesta
\end{itemize}

Cada uno de estos módulos cuenta a su vez con subcomponentes sobre los cuales hay mayor o menor consenso, que determinan, en definitiva, la performance del sistema concreto implementado. 

El módulo de procesamiento de la pregunta tiene como subcomponente principal un Clasificador de Preguntas (Ver \allref{subsec:qc} y \allref{sec:stanford-qc}) y puede incorporar otros componente como identificadores del \sq{foco} (\textit{focus}) y del tipo esperado de la respuesta (\textit{answer type}), y también puede incluirse en este módulo un componente encargado de expandir o reformular la pregunta para optimizar los resultados del módulo de procesamiento de documentos ( este subcomponente también se suele considerar como parte del módulo de procesamiento de respuestas). Este último subcomponente tiene diferentes nombres en la literatura (\textit{query reformulation}, \textit{query generation}, \textit{query expansion}, etc).

En el núcleo del módulo de procesamiento de documentos, por lo general, hay uno o más índices invertidos (como Lucene o Indri) como el descripto en \allref{subsec:indice-invertido}, o bien accesos a un buscador web, o, más en general, estructuras de information retrieval. Como vimos (\ref{subsec:metricas-ir}), existen dos métricas fundamentales a la hora de evaluar el funcionamiento de un módulo de information retrieval: precisión y recall. La precisión es la proporción de documentos relevantes devueltos sobre el total documentos devueltos por el módulo, mientras que el recall es la proporción de documentos relevantes devueltos sobre el total de documentos relevantes existentes en la base de conocimientos. Estas dos métricas suelen comportarse una con la otra como un trade-off, es decir: suelen aparecer escenarios de diseño en los que se deberá optar por priorizar una o bien la otra. Considerando esto, existe una tercera métrica extendida, la medida $F$, que busca parametrizar este compromiso. 

En el contexto de uso de los sistemas de QA, la métrica que se busca maximizar para el módulo de procesamiento de documentos es el recall, es decir, no interesa que el módulo no recupere documentos no relevantes siempre y cuando recupere todos los documentos relevantes existentes que pueda. El motivo es simple: se espera que el análisis lingüístico fino del módulo de procesamiento de la respuesta sea capaz de eliminar texto irrelevante en la misma o en mejor medida que el sistema de information retrieval subyacente al módulo de documentos. Por otro lado, si al recuperar los documentos se filtra una respuesta válida como irrelevante, entonces el resto del pipeline vera sus probabilidades de éxito mermadas (sino directamente anuladas).  La expansión/reformulación de la pregunta del paso anterior apunta, justamente, a generar una query que priorice el recall sobre la precisión. 

Otras tareas que se realizan en el módulo de procesamiento de documentos es la división de los documentos en parágrafos, el filtrado de documentos o parágrafos irrelevantes y el ranking de los resultados. Si bien los sistemas de information retrieval suelen rankear sus resultados, este ranking puede resultar no del todo adecuado para la tarea en cuestión y mejorable teniendo en cuenta la información disponible. 

Finalmente, el módulo de procesamiento de respuestas introduce el valor agregado que distingue un sistema de IR típico y un sistema de QA propiamente dicho. Este módulo es el menos estandarizado de los tres y tiene diferentes enfoques más o menos elaborados en la literatura, pero el grado de consenso sobre la viabilidad del uso de un grupo de técnicas fijas sobre otras no ocurre como, por ejemplo, en el uso de un clasificador de preguntas en el módulo de procesamiento de preguntas o de un índice de búsquedas en el módulo de procesamiento de documentos. Típicamente, el módulo consta de tres momentos: a partir de los parágrafos o documentos rankeados generados por el módulo de documentos se busca identificar oraciones o respuestas candidatas mediante distintas técnicas. Para estas respuestas candidatas identificadas, se extrae la información concreta que responde a la pregunta utilizando heurísticas basadas en las etiquetas obtenidas para la pregunta (clase de pregunta, foco, tipo de respuesta, etc). Finalmente, se implementa algún mecanismo de corroboración o de validación de respuestas, generando una respuesta final.

A continuación nos extenderemos con más detalle sobre los tres módulos propuestos señalando algunos enfoques usuales, usando como guía los papers \cite{QA-survey}, \cite{QA1}, \cite{QA2}, \cite{QA3} y \cite{PASSAGE1}.

\subsubsection*{Módulo de procesamiento de la pregunta}
Este módulo recibe como input una pregunta formulada en algún lenguaje natural y genera como output una representación de la misma que resulte útil para los módulos siguientes. Esto suele realizarse agregando labels o etiquetas (ver detalle de \dq{Features}, en \allref{sec:terminologia}) a la pregunta completa y a sus distintos términos.

La etiqueta principal generada es el tipo de pregunta (\textit{question type}) y para obtenerlo se utiliza un clasificador como el que describimos en (\ref{subsec:qc})· Lamentablemente, la clasificación de preguntas no está desarrollada para otros idiomas que el inglés, al menos, no con el mismo grado de precisión y visibilidad que, por ejemplo, el clasificador de Stanford \cite{QC1} \cite{QC2}. Como vimos anteriormente, el principal problema a la hora de clasificar una pregunta es la ambigüedad intrinseca de ciertas clases de preguntas. 

A partir del tipo de pregunta se define el tipo de respuesta esperada (\textit{answer type}). No existe un algoritmo automático estándar difundido para generar el tipo de respuesta esperado. El enfoque general encontrado es la aplicación de un mappeo simple basado en reglas, desde el tipo de pregunta generado por el clasificador a una serie de tipos de respuestas predefinido para el dominio de problema.

Otro análisis interesante pero, según nuestra investigación, no sistematizado, es el análisis del \dq{foco} (\textit{question focus}). En \cite{QA3} se define el foco como una palabra o secuencia de palabras que indican qué se está preguntando. Por ejemplo, para la pregunta \dq{¿Quién fue nombrado presidente de Argentina en  1983?} el foco sería \dq{presidente}). {\color{red} VER MI RESUMEN DE QA-SURVER, COMENTARIO PERSONAL}. El concepto aparece también en \cite{WATSON1} pero no encontramos detalles sobre los mecanismos técnicos implicados en la extracción y todo parece señalar a una serie de heurísticas de procesamiento lingüístico escritas a mano. 

Otros análisis que se realizan sobre la pregunta son el POS-tagging (Ver \allref{subsec:pos}), el reconocimiento de entidades nombradas (NER, ver \allref{subsec:ner}) y la búsqueda de dígitos u otros patrones útiles para el contexto. 

Finalmente, se ejecuta la reformulación o expansión de la pregunta para genera un input para el módulo de information retrieval tal que maximice el recall. 
La idea guía es la extracción de keywords relevantes y existen diferentes enfoques, usando los pos-tags y el resultado del ner. También pueden utilizarse recursos léxicos como Wordnet para generar sinónimos de términos importante. 

En \cite{QA1} y \cite{QA3} se propone una heurística basada en 8 reglas para implementar este paso. Las reglas propuestas son:

\begin{itemize}
\item {\color{red} Fill me neddy}
\end{itemize}

\subsubsection*{Módulo de procesamiento de documentos}
El input del módulo de procesamiento de documentos es la pregunta reformulada según describimos en los párrafos inmediatamente anteriores y su output es una lista de documentos (o parágrafos) ordenados según la factibilidad de que contengan la respuesta a la pregunta. El módulo consta, en general, de uno o más sistemas de information retrieval (entre los cuales puede estar la web indexada) y de submodulos de filtrado, fragmentación y reordenamiento. Sobrelos sistemas de information retrieval es importante notar que suelen usarse sistemas tradicionales y no aquellos basados en distancias coseno, como el utilizado, por ejemplo, en latent semantic analysis (LSA). Este último enfoque, que demostró ser útil a la hora de eliminar o reducir problemas de {\color{red} qué resuelve LSA} en information retrieval, pero en QA es importante que las keywords de búsqueda mismas estén presentes en los documentos relevantes y no que sean solamente \dq{semánticamente similares}. Dada la gran cantidad de documentos que se espera que retorne el subsistema de information retrieval (como señalamos, se prioriza el recall y no la precisión), existe un paso de filtrado (en \cite{WATSON1} se lo nombra como \sq{soft filtering}) en el cual: 1\textsuperscript{o} se reduce la cantidad de documentos a retornar por el módulo en general y 2\textsuperscript{o} se reduce la cantidad de texto en cada documento. El principio que guía este proceso es la idea de que las palabras clave de la pregunta deben aparecer, de alguna manera, cercanas entre sí. De este modo, es posible buscar por estas keywords en los documentos y quedarse solo con los $N$ parágrafos que contengas las keywords, desechando el resto del texto

{\color{red}Implementación de LASSO??}

Finalmente, se procede a ordenar los parágrafos / documentos filtrados. Una implementación posible de este ordenamiento es utilizar radix sort, considerando los siguientes tres scores. {\color{red} INCONCLUSO}


\subsubsection*{Módulo de procesamiento de la respuesta}
Finalmente, el modulo de procesamiento de respuestas es el encargado de generar una respuesta a partir del output de los otros dos módulos. Para ello debe identificar la respuesta dentro de los parágrafos, extraer la fracción de texto que responde puntual y concretamente a la pregunta y, finalmente, validar su corrección. 

Para identificar la respuesta se utilizan distintas heurísticas, utilizando el tipo de respuesta esperado cuando es posible. Sin embargo, el tipo de respuesta esperado no siempre es explícito en la pregunta o en la respuesta, por lo que se utilizan diferentes reglas basadas en pos-tags, ners, en un parser específico para el reconocimiento de las respuestas candidatas.

Para las respuestas candidatas se extrae luego el o los términos concretos de la respuesta final {\color{red} Por ejemplo: }. Existe sobre este punto heúristicas más o menos avaladas basadas en distancias entre keywords, número de keywords presentes y otras métricas similares (ver, por ejemplo, {\color{red} Qanus, scorers}. Típicamente, si no se encuentra ninguna coincidencia confiable, los sistemas de QA productivos recaen en la devolución de los primeros $n$ parágrafos mejor rankeados por el módulo de documentos. 

Como señalamos anteriormente, en las competencias TREC hasta 2001 se permitía devolver una lista de 5 respuestas, mientras que desde el 2002 se exige una única respuesta.

Finalmente, hay distintas formas de validar la corrección o de estimar un grado de confianza en la corrección de una respuesta. En primer lugar, pueden usarse diferente corpus especificos {\color{red} INCONCLUSO}.

\subsection{QA como interfaz a una base de datos}
\label{subsec:closed-domain}
\falta
El enfoque a la hora de encarar una solución de question answering sobre dominios cerrados (específicos) difiere sustancialmente de los enfoques sobre dominios abiertos. Un dominio de conocimientos general fuerza datos en texto plano, documentos escritos en lenguaje humano, tal como vimos en la sección anterior. La razón de esto es que no existen bases de datos, o, en general, repositorios de datos estructurados acerca de... todo. En sistema open domain, sin embargo, es posible incorporar bases de conocimiento estructuradas como complemento de otras no estructuradas. Cuando coexisten ambos tipos de datos, como vimos, hablamos de sistemas híbridos. 

Los usos que puede hacer un sistema híbrido de sus bases de información estructurada son varias. Un uso típico es responder preguntas sobre dominios específicos, esto es, un sistema puede utilizar bases de conocimiento estructuradas y algoritmia específica para ciertos temas concretos (por ejemplo: Liga Americana del año 1958) y para ciertos tipos de preguntas (¿cuál fue el resultado del partido...?), mientras que para el resto de las preguntas recae en métodos de dominio abierto. Estos "subsistemas" son más costosos y dificiles de construir, pero, en general, logran mejores resultados para las preguntas que acepta (y brindan una estimación de la confiabilidad de la respuesta bastante buena). Otro uso típico es verificar el tipo de una respuesta extraido de un curpus plano, contra ontologías como Freebase, Yago o dbPedia \cite{WATSON2}, por ejemplo, verificando que la respuesta generada para \dq{¿Quién fue el presidente de...?} sea una persona y, mejor, un presidente. 

Los sistemas de dominio cerrado puros, por otro lado, tienen interés en diferentes ámbitos en los que se dispone de información estructurada de utilidad para personas sin los conocimientos técnicos necesarios para escribir una consulta en el lenguaje formal que brindan las bases de datos como interfaz de consultas.

Para ambos casos de uso de los modelos de dominio cerrado (sistemas híbridos y de dominio cerrado puros) el enfoque algorítmico consiste en traducir la pregunta formulada en un lenguaje natural a una consulta del lenguaje formal (potencialmente: SQL, SPARQL o similar). Dado que la cantidad de "cuestiones" sobre las que se puede responder es siempre acotada, el modelo estándar de dominio cerrado sobre datos estructurados busca traducir la pregunta formulada en lenguaje humano a una consulta acerca de algún aspecto de la base de datos. Debe, además poder decidir si una pregunta tiene sentido en el contexto de la información de la que dispone. 

En \cite{QADB1} los investigadores proponen un modelo teórico simplificado para enfocar este problema que les permite definir, para una pregunta cualquiera $q$, su \textit{tratabilidad semántica} en el contexto de una base de datos estructurada. A continuación introduciremos una serie de definiciones que forman este modelo  para ilustrar el enfoque general al problema. 

En el contexto de una base de datos relacional, podemos distinguir los siguientes tres tipos de elementos: las \textbf{relaciones}, los \textbf{atributos} y los \textbf{valores}. Por relación nos referimos, en principio, a colecciones (como universidades, películas, actores, inventarios, etc), por atributo a una propiedad de los elementos de esa colección (dirección de una universidad, año de estreno de una película) por valor a los datos concretos (\dq{Figueroa Alcorta 2263}, \dq{2005}). Para evitar confusiones con el uso técnico de teoría de bases de datos del término \dq{relación} y las relaciones entre entidades (por ejemplo: la relación \textit{trabajar-en} entre docentes y universidades), a partir de aquí utilizaremos el término \textit{colección}.

Sobre estos tres tipos de elementos que puede tener una base de datos, se formaliza la noción de \textbf{compatibilidad} como sigue: 
\begin{itemize}
  \item Un valor es compatible con su atributo. Por ejemplo, \dq{Figueroa Alcorta 2263} es compatible con \textit{dirección}. 
  \item Un valor es compatible con su colección. Por ejemplo, \dq{Figueroa Alcorta 2263} es compatible con \textit{universidades}.
  \item Un atributo es compatible con su colección. Por ejemplo, \textit{dirección} es compatible con \textit{universidades}.
  \item Cada atributo dispone de un conjunto de qwords compatibles.
\end{itemize}

Con respecto al cuarto bullet, recordemos que las \textbf{qwords} son las palabras típicas con las que comienza una pregunta (quién, qué, cómo, cuándo, dónde, etc para el castellano, why, where, who, which, etc para el inglés). Mientras que las primeras tres cláusulas son formales, la compatibilidad de un atributo con ciertas qwords debe ser definida. Un ejemplo de compatibilidad entre una qword y un atributo es \textit{dónde} con \textit{dirección}. En la definición de tratabilidad semántica que expondremos en breve, se pide que en la pregunta exista una qword, para limitar el espectro de preguntas tratables. 


A la hora de procesar la pregunta, de los tokens de $q$ se distinguen, primero, los \textbf{marcadores sintácticos}, que son palabras funcionales que no aportan información semántica (por lo general, stopwords como \sq{a} y \sq{de}, etc). Al resto de los tokens se los busca relacionar con elementos de la una base de datos genérica $D$ mediante una función de traducción definida para el caso específico. Por ejemplo, se espera de una función de traducción de tokens en elementos que los pares de tokens \{experiencia, requerida\} y \{experiencia, exigida\} coincidan con el atributo \sq{Experiencia Requerida}. 

Para construir la tratabilidad semántica de $q$ en $D$ utilizaremos el concepto de \textbf{conjunto de traducción de tokens}, al cual notaremos como \tradqd \footnote{Popescu llama a esta este conjunto de tokens -en realidad, stems- como \textit{token} , palabra que tenemos tomada y nos forzó a cambiar el nombre}, como cualquier conjunto de tokens de $q$ que mapee con un elemento $E$ de la $D$ según la función de traducción. Los tres diferentes tipos de elementos determinan tres tipos de \tradqd s distintas (traducciones de relaciones, de atributos y de valores). Una \textbf{traducción completa} de $q$ es un conjunto de \tradqd s en los que cada token de $q$ aparece una y solo una vez. Ilustremos todos estos conceptos con un ejemplo: consideremos una base de datos $D$ con información sobre universidades y la siguiente pregunta $q$: \dq{¿Cuántos años tienen de docentes en estado activo que trabajan en la UBA}, una traducción completa es la siguiente: $\{\{\text{cuántos}, \text{años}, \text{tienen}\}_{attr=edad}, \{\text{docentes}\}_{rel=profesores}, \newline \{estado\}_{attr}, \{activo\}_{val}, \{trabajan\}_{attr=universidad}, \{UBA\}_{val=\text{Universidad de Buenos Aires}}\}$. En esta traducción, quedan de lado las stopwords \{de, en, que, en, la\}, y se agrupan los tres primero tokens en el \tradqd  $\{\{\text{cuántos}, \text{años}, \text{tienen}\}$ que mapea al atributo de $D$ \textit{edad}, y se mapea el \tradqd  unitario \{\dq{docentes}\} con la tabla \textit{profesores}, etc. 

Una traducción completa de $q$ se considera una \textbf{traducción válida} en el contexto de una base de datos $D$ si cumple las siguientes tres condiciones:
\begin{enumerate}
  \item Cada \tradqd  $t_i$ mapea con un único elemento $E$ de $D$
  \item Cada \tradqd  mapeada a un atributo $t_a$ corresponde a una única \tradqd  mapeada a un valor $t_v$. Esto significa:
    \begin{itemize}
      \item El atributo en $D$ al que mapea $t_a$ y el valor en $D$ al que mapea $t_v$ son compatibles.
      \item $t_a$ y $t_v$ están sintácticamente relacionados en el parse-tree de $q$
    \end{itemize}
  \item Cada \tradqd  mapeada a una colección en $D$ ($t_r$) corresponde con una \tradqd  de atributo ($t_a$) o bien con una \tradqd  de valor ($t_v$). Esto significa:
      \begin{itemize}
      \item La colección en $D$ a la que mapea $t_r$ y el elemento en $D$ al que mapea $t_a$ (o $t_v$) son compatibles 
      \item $t_r$ y $t_a$ (o $t_v$) están sintácticamente relacionados en el parse-tree de $q$
    \end{itemize}
\end{enumerate}

Finalmente, una pregunta $q$ se dice \textbf{semánticamente tratable} en el contexto de una base de datos $D$ si existe un mapeo válido con qtokens distintos y en el que al menos uno de sus \tradqd  incluye una qword.

En los casos de uso de un sistema de question answering de dominio cerrado resulta fundamental decidir si una pregunta puede o no responderse. El costo de rechazar una pregunta o bien pedir una re-especificación al usuario es mucho menor que el de mal interpretar activamente una pregunta y generar una respuesta inválida. En este contexto, una definición operacional de tratabilidad semántica de una pregunta en el contexto de una base de datos como la enunciada recién resulta esencial. Por otro lado, es importante notar las limitaciones del modelo teórico recién expuesto. En principio, la navegación a través del modelo de datos no está contemplada y el manejo de relaciones se limita a relaciones de un grado. Esta noción de tratabilidad semántica se basa en la observación cierta de que muchas preguntas expresadas en lenguaje natural especifican pares de atributo/valor o especifican solo valores, dejando el atributo implícito. Si bien es cierto que existen formulaciones de preguntas complejas de preguntas simples y también preguntas esencialmente complejas de formular, una gran cantidad de ellas tiene un formato muy sencillo como:
\dq{¿Dónde está X?}, \dq{¿Quién está relacionado con X (de alguna manera)?}, \dq{¿Cuál es el X con atributos Y relacionado con Z?}, etc. 

En la proxima sección veremos un caso de éxito de un proyecto de question answering híbrido implementado por IBM y en el próximo capítulo describiremos la implementación de un modelo de question answering de dominio cerrado en el que utilizamos este marco como orientación. 


\subsection{IBM-Watson}
\label{subsec:ibm-watson}
\falta
Watson\cite{WATSON1}\cite{WATSON2} es un sistema diseñado por IBM con el objetivo de competir en
tiempo real en el programa de televisión estadounidense Jeopardy,
logrando resultados del nivel de los campeones humanos de este
programa.

El proyecto demoró 3 años de investigación, en los cuales se
logró obtener la performance esperada (nivel humano experto) en
cuanto a precisión, confiabilidad y velocidad, logrando derrotar a
dos de los hombre con mayores récords históricos del show en un
programa en vivo\footnote{En esta url está disponible el programa en el que el sistema vence a sus competidores humanos: \url{http://www.youtube.com/watch?v=WFR3lOm_xhE}} en febrero de 2011.

El objetivo del \ proyecto puede considerarse una extensión de lo que
fue Deep Blue, el sistema que logró el nivel de los expertos humanos
en el ajedrez, porque buscó superar un reto que significativo y
visible del campo de la Inteligencia Artificial tanto para la comunidad
científica como para la sociedad en general:
{\textquotedblleft}?`puede un sistema computacional ser diseñado para
competir con los mejores hombre en alguna tarea que requiera altos
niveles de inteligencia humana y, si es el caso, que clase de
tecnología, algoritmos e ingenieria se
requiere?{\textquotedblright}\footnote{Traducción propia de
un fragmento de \cite{WATSON1}, p. 2}

Watson es la implementación específica para participar en este
programa de una arquitectura más genérica de question answering,
DeepQA, que da el nombre al proyecto de la corporación. Esta
arquitectura es de construcción reciente y ejemplifica perfectamente la complejidad del problema de
QA de dominio abierto e incorpora tecnologías de punta de distintos
dominios de ciencias de la computación, y de IA en particular:
information retrieval, natural language processing, knowledge
representation and reasoning, machine learning e interfaces humano -
computadora. En el transcurso de esta tesis, IBM lanzó el programa
\sq{Watson Ecosystem} (en noviembre de 2013) que promete la utilización
de tecnología de punta para aplicaciones creadas por la comunidad\footnote{
Announcing the IBM Watson Ecosystem Program: \url{http://www-03.ibm.com/innovation/us/watson/}}.

\subsubsection*{El problema}

Watson debe realizar tareas como parsing, question classification,
question descomposition, automatic source adquisition and evaluation,
entity and relation detection, logical form generation, knowledge
representation and reasoning manteniendo ciertos atributos de calidad
bastante exigentes derivados de la naturaleza del show. Estas
restricciones son:

\begin{itemize}
\item Confiabilidad de la respuesta: \newline
Jeopardy tiene tres participantes con un pulsador y el que desee
responder debe pulsar antes que los demás. Además, existe una
penalización por respuestas incorrectas, por lo que es esencial que
el sistema pueda determinar la confiabilidad de la respuesta obtenida a
fin de optar por responder o no responder.
\item Tiempos de respuesta: \newline
La confiabilidad de la respuesta, o al menos una estimación, debe
calcularse antes de que pase el tiempo para decidir responder (6
segundos) y también de que otro participante oprima su pulsador
(menos de 3 segundos).
\item Precisión:\newline
El tipo de respuestas que se dan en el show suelen ser respuestas
exactas (por ejemplo: solamente un nombre, un número o una fecha,
etc). 
\end{itemize}

\bigskip

El sistema cuenta con varios componentes heurísticos que estiman
ciertos features y grados de confiabilidad para diferentes respuestas,
los cuales son evaluados por un sistema general que sintetiza un grado
de confiabilidad para una respuesta final y determina así si
responder o no responder. 

El programa consta de un tablero con 30 pistas (o preguntas) organizadas
en seis columnas, cada una de las cuales es una categoría. Las
categorías van desde temas acotados como
{\textquotedblleft}historia{\textquotedblright} o
{\textquotedblleft}ciencias{\textquotedblright} hasta temas más
amplios como {\textquotedblleft}cualquier cosa{\textquotedblright} o
{\textquotedblleft}potpourri{\textquotedblright}. Watson intenta
respuestas sobre varias hipótesis de dominio y verifica en cual de
ellos se logran respuestas de mayor confiabilidad. 

Por otra parte, el grueso de las preguntas de Jeopardy son del tipo
\textit{factoid}, esto es, preguntas cuya respuesta esta basada en
información fáctica acerca de una o más entidades individuales.


\bigskip

Por ejemplo:

Categoría: Ciencia General

Pista: Cuando es impactado por electrones, un fósforo emite energía
electromagnética de esta forma

Respuesta: Luz (o fotones)


\bigskip

A su vez, existen ciertos tipos de pistas que requieren un enfoque
particular, por ejemplo, pistos que constan de dos subpistas muy
débilmente relacionadas, o problemas matemáticos formulados en
lenguaje humano, o problemas de fonética, etc, que no pueden ser
simplemente dejados de lado porque, si bien tiene poca probabilidad de
aparición, cuando aparecen lo hacen en bloque y pueden arruinar el
juego de Watson. Se acordó con la productora del programa, sin
embargo, dejar de lado preguntas audiovisuales (aquellas que presentan
una imagen o un audio y requieren interpretarlo) y preguntas que
requieren instrucciones verbales del presentador.


\bigskip

Para determinar el dominio de conocimiento, los investigadores
analizaron 20000 preguntas, extrayendo su LAT (lexical answer type, o
tipo léxico de respuesta). El LAT se define como una palabra en la
pista que indica el tipo de la respuesta esperado. Por ejemplo, para la
pista {\textquotedblleft}Investanda en 1500{\textquoteright}s para
agilizar el juego, este movimiento involucra dos
piezas{\textquotedblright} el LAT es
{\textquotedblleft}movimiento{\textquotedblright}. Menos del 12\% de
las pistas no indicaba explícitamente ningún LAT, usando palabras
como {\textquotedblleft}esto{\textquotedblright} o
{\textquotedblleft}eso{\textquotedblright}. En estos casos, el sistema
debe inferir el tipo de respuesta del contexto. Del análisis de estas
20000 pistas se reconocieron 2500 tipos léxicos distintos, de los
cuales los 200 más frecuentes no llegaban a cubrir el 50\% del total
de pistas. Esto implica que un approach estructurado (orientado por el
tipo de respuesta), si bien resulta útil para algunos tipos, no es
suficiente para abordar el problema completo.

\subsubsection*{Métricas}

Las métricas de resultados, además del tiempo de respuesta, son la
\textit{precisión} (preguntas contestadas correctamente / preguntas
contestadas) y el \textit{porcentaje de respuestas dadas }(preguntas
contestadas / total de preguntas). Mediante la configuración de un
threshold de \textit{confiabilidad} pueden obtenerse distintas
estrategias de juego: un umbral bajo repercutirá en un juego más
agresivo, incrementando la proporción de respuestas contestadas,,
pero disminuyendo su precisión, mientras que un umbral alto
determinará un juego conservador, con menos respuestas dadas pero
mayor precisión en las mismas. Es un clásico escenario de trade-off
entre dos atributos de calidad. Un buen sistema de estimación de
confiabilidad implica una mejora general del sistema, aún cuando el
módulo de generación de respuestas permanezca idéntico.


\bigskip

En el show, el porcentaje de respuestas dadas depende de la velocidad
con la que se llega a presionar el pulsador, lo cual sólo interesa
para el dominio de QA como una restricción temporal. 


\bigskip

Mediante análisis numérico, los investigadores determinaron que los
campeones de Jeopardy lograban tomar entre el 40\% y el 50\% de las
preguntas y, sobre ellas, lograban una precisión de entre el 85\% y
el 95\%, lo que determinaba una barrera de performance bastante
exigente en lo que respecta a QA.


\bigskip

\subsubsection*{Baseline}

El equipo de IBM intentó utilizar dos sistemas consolidados en QA y
adaptarlos al problema \ de Jeopardy. \ El primero fue PIQUANT
(Practical Intelligent Question Answering Technology), un sistema
desarrollado por IBM en conjunto con el programa del gobierno
estadounidense AQUAINT y varias universidades, que estaba entre los
mejores según la TREC (Text Retrieval Conference), una autoridad en
el área. PIQUANT consta de un pipeline típico (véase QANUS) con
tecnología de punta, logrando un rango del 33\% de respuestas
correctas en las evaluaciones TREC-QA. Los requerimientos de la
evaluación de TREC son muy distintos de los de Jeopardy: TREC ofrece
un corpus de conocimiento relativamente pequeño (1M de documentos) de
donde las respuestas deben ser extraídas y justificadas, el tipo de
preguntas de TREC son menos complejas a nivel ling\"uístico que las
de Jeopardy y la estimación de confiabilidad no resulta una métrica
importante (dado que no hay penalización por respuestas incorrectas).
Además, los sistemas tienen permitido acceder a la web y las
restricciones temporales son, por mucho, más amplias (por ejemplo:
una semana para responder 500 preguntas). En Jeopardy, además de las
restricciones ya mencionadas, un requerimiento fue que el sistema
trabaje sobre datos locales y no acceda a la web en tiempo real. El
intento de adaptar PIQUANT al problema de Jeopardy dio pésimos en
comparación con los necesarios: 47\% de precisión sobre el 5\% de
respuestas con mayor confiabilidad y 13\% de precisión en general. 

Por otro lado, el equipo intentó adaptar el sistema OpenEphyra
(véase OpenEphyra), un framework open-source de QA desarrollado en
CMU (Carnegie Mellon University) basado en Ephyra (no libre),
diseñado también para la evaluación TREC. OpenEphyra logra un
45\% de respuestas correctas sobre el set de datos de evaluación TREC
2002, usando busqueda web. La adaptación resultó aún peor que la
de PIQUANT (con menos del 15\% de respuestas correctas y una mala
estimación de la confiabilidad). 

Se probaron dos adaptaciones de estos sistemas. una basada en
búsquedas de texto puro y otra basada en reconocimiento de entidades.
En la primera, la base de conocimiento se modeló de manera no
estructurada y las preguntas se interpretaron como términos de una
query, mientras que en la segunda se modeló una base de conocimientos
estructurada y las preguntas se analizaron semánticamente para
reconocer entidades y relaciones, para luego buscarlos en la base.
Comparando ambos enfoques en base al porcentaje de respuestas dadas, el
primero dio mejores resultados para el 100\% de las respuestas,
mientras que la confiabilidad general era baja; por otro lado, el
segundo enfoque logró altos valores de confiabilidad, pero sólo en
los casos en que efectivamente logra identificar entidades. De aquí
se infiere que cada enfoque tiene sus ventajas, en el dominio de
problemas apropiado.

\subsubsection*{La arquitectura DeepQA}
\label{subsec:deep-qa}
Los intentos de adaptación iniciales, como vimos, no dieron
resultados, así como tampoco sirvieron las adaptaciones de algoritmos
de la literatura científica, los cuales son realmente difíciles de
sacar de su contexto original y de las evaluaciones sobre las cuales
fueron testeados. Este problema, veremos -por ejemplo, con QANUS y
Reverb- , se repitió en nuestro proyecto. Como conclusión de estos
intentos frustrados, el equipo de IBM entendió que una arquitectura
de QA no debía basarse en sus componentes concretos sino en la
facilidad para incorporar nuevos componentes y para adaptarse a nuevos
contextos. Así surgió DeepQA, la arquitectura de base, de la cual
Watson es una instancia concreta para un contexto particular (con
requerimientos de alta precisión, buena estimación de
confiabilidad, lenguaje complejo, amplitud de dominio y restricciones
de velocidad). DeepQA es una arquitectura de computo paralelo,
probabilistico, basado en recopilación de evidencia y scoring. Para
Jeopardy se utilizaron más de 100 técnicas diferentes para analizar
lenguaje natural, identificar y adjudicar valor a fuentes de
información, encontrar y generar hipótesis, encontrar y rankear
evidencias y mergear y rankear hipótesis en función de esta
evidencia. La arquitectura sirvió para ganar Jeopardy, pero también
se adaptó a otros contextos como la evaluación TREC, dando
resultados mucho mejores que sus predecesores. Los principios de
diseño subyacentes de la arquitectura son:

\begin{itemize}
\item Paralelismo masivo\newline
Para evaluar distintas hipótesis en distintos dominios con poco
acoplamiento.
\item Pervasive confidence estimation:\newline
Ningún componente genera la respuesta final, sino que da una serie de
features y grados de confiabilidad y evidencia para distintas
hipótesis, que luego son sintetizados.
\item Integrate shallow and deep knowledge:
\end{itemize}

\bigskip

A continuación, enumeraremos la lista de pasos que sigue el sistema
para obtener la respuesta a una pregunta:

\subsubsection*{Adquisición de contenidos}

El primer paso de DeepQA es la adquisición de contenidos. Este paso es
el único que no se realiza en tiempo de ejecución y consiste en
crear la base de conocimiento en la cual el proceso final buscará la
respuesta a la pregunta, combinando subprocesos manuales y
automáticos. 

En principio se caracteriza el tipo de preguntas a responder y el
dominio de aplicación. El análisis de tipos de preguntas es una
tarea manual, mientras que la determinación del dominio puede
encararse computacionalmente, por ejemplo, con la detección de LATs
que señalamos antes. Dado el amplio dominio de conocimientos que
requiere Jeopardy, Watson cuenta con una gran cantidad de
enciclopedias, diccionarios, tesauros, artículos académicos y de
literatura, etc. A partir de este corpus inicial, el sistema busca en
la web documentos relevantes y los relaciona con los documentos ya
presentes en el corpus. 

Además de este corpus de documentos no estructurados, DeepQA maneja
contenidos semi-estructurados \ y estructurados, incorporando bases de
datos, taxonomías y ontologías como dbPedia, Wordnet y las
ontologías de Yago. 

\subsubsection*{Análisis de la pregunta}

El primer paso en run-time es el análisis de la pregunta. En este paso
el sistema intenta entender qué es lo que la pregunta está
preguntado y realizar los primeros análisis que determinan cómo
encarará el procesamiento el resto del sistema. Watson utiliza
shallow parses, deep parses, formas lógicas, pos-tags,
correferencias, detección de entidades nombradas y de relaciones,
question classification, además de ciertos análisis concretos del
domiento del problema.

En este proceso se clasifica el tipo de la pregunta (los tipos están
determinados por el show: puzzles, matemáticos, etc). También se
busca el tipo de respuesta esperada, dónde los tipos manejados son
por Watson son los LATs extraídos de las preguntas de ejemplo. El LAT
determina el {\textquotedblleft}tipo{\textquotedblright} de la
respuesta, que clase de entidad \textit{es} la respuesta (una fecha, un
hombre, una relación, etc). El equipo de IBM intentó adaptar
distintos algoritmos de clasificación preexistentes, pero después
de intentar entrenarlos para el dominio de tipos de Jeopardy, llegaron
a la conclusión de que su eficacia era dependiente del su sistema de
tipos default, y que la mejor forma de adaptación era mappear su
output a los tipos utilizados por Watson (un enfoque similar fue
utilizado en esta tesis con respecto al clasificador de Stanford). Otra
anotación importante es el
{\textquotedblleft}foco{\textquotedblright} de la pregunta, la parte de
la pregunta tal que si se la reemplaza por la respuesta, la pregunta se
convierte en una afirmación cerrada.

Por ejemplo, para {\textquotedblleft}El hombre que escribió Romeo y
Julieta{\textquotedblright}, el foco es {\textquotedblleft}El hombre
que{\textquotedblright}. Este fragmento suele contener información
importante sobre la respuesta y al reemplazarlo por una respuesta
candidata se obtiene una afirmación fáctica que puede servir para
evaluar distintos candidatos y recolectar evidencia. Por ejemplo,
reemplazando por distintos autores y verificando que la oración
resultante esté presente en el corpus.

Por otro lado, muchas preguntas involucran relaciones entre entidades y,
más puntualmente, tienen una forma sujeto-verbo-objeto. Por ejemplo,
tomando la pista anterior, podemos extraer la relación
\textit{escribir(x, Romeo y Julieta)}. La amplitud del dominio de
Jeopardy hace que la cantidad de entidad y de relaciones entre
entidades sea enorme, pero esto empeora aún más al considerar las
distintas formas de expresar la misma relación. Por eso, Watson
sólo logra encontrar directamente una respuesta mediante
reconocimiento de entidades y relaciones sobre el 2\% de las pistas. En
general, este tipo de enfoque es útil sobre dominios más acotados,
mientras que la detección de relaciones como approach general a un
problema de question answering de dominio amplio es un área de
investigación abierta. 

Una particularidad ya señalada de las preguntas de Jeopardy son las
pistas con subpistas no relacionadas. Para atacar este problema, Watson
genera distintas particiones y resuelve todas en paralelo, sintetizando
las respuesta de cada partición generada mediante algoritmos ad-hoc
de ponderación de confiabilidad y otras características.

\subsubsection*{Generación de hipótesis}

El tercer paso (segundo en run-time) es la generación de hipótesis:
tomando como input el resultado del paso anterior se generan respuestas
candidatas a partir de la base de conocimiento offline. Cada respuesta
candidata reemplazada por el foco de la pregunta es considerada una
hipótesis, que el sistema luego verificará buscando evidencias y
adjudicando un cierto grado de confiabilidad.

En la búsqueda primaria de respuestas candidatas, se busca generar
tantos pasajes como sea posible. El resultado final obtenido revela que
el 85\% de las veces, la respuesta final se encuentra entre los
primeros 250 pasajes devueltos por la búsqueda primaria. La
implementación utiliza una serie variada de técnicas, que incluyen
diferentes motores de búsqueda de textos (como Indri y Lucene),
búsqueda de documentos y de pasajes, búsquedas en bases de
conocimiento estructuradas como SPARQL con triple store y la
generación de mutiples queries a partir de una sola pregunta. La
búsqueda estructurada de triple stores depende del reconocimiento de
entidades y relaciones del paso anterior.

Para un número pequeño de LATs, se definió una suerte de conjunto
de entidades fijas (por ejemplo: países, presidentes, etc). Si la
respuesta final no es retornada en este paso, entonces no hay
posibilidad de obtenerla en los siguiente. Por eso se prioriza el
recall sobre la precisión, con el supuesto de que el resto del
pipeline logrará filtrar la respuesta correcta correctamente. Watson
genera varios cientos de hipótesis candidatas en este paso.


\bigskip

\subsubsection*{(Soft filtering)}

Para optimizar recursos, se realiza un filtrado liviano de respuestas
antes de pasar a la recopilación de evidencia y al scoring de
hipótesis. Un filtrado liviano es, por ejemplo, comprobar similaridad
de la respuesta candidata con el LAT esperado de la respuesta. Aquellas
hipótesis que pasan el filtro pasan al siguiente proceso, que realiza
un análisis más exhaustivo.


\bigskip

\subsubsection*{Recuperación de evidencias y scoring de pasajes}

Para recuperar evidencias se utilizan varios algoritmos. Uno
particularmente útil es buscar la hipótesis candidata junto con las
queries generadas por la pregunta original, lo que señala el uso de
la respuesta en el contexto de la pregunta. \ Las hipótesis con sus
evidencias pasan al siguiente paso, dónde se les adjudica un score. 

El proceso de scoring es donde se realiza la mayor parte del análisis
más fuerte a nivel computacional. DeepQA permite la incorporación
de distintos Scorers, que consideran diferentes dimensiones en las
cuales la hipótesis sirve como respuesta a la pregunta original. Esto
se llevó a cabo definiendo una interfaz común para los scorers.
Watson incorpora más de 50 componentes que producen valores y
diferentes features basados en las evidencias, para los distintos tipos
de datos disponibles (no estructurados, semi estructurados y
estructurados). Los scorers toman en cuenta cuestiones como el grado de
similaridad entre la estrurctura de la respuesta y de la pregunta,
relaciones geoespaciales y temporales, clasificación taxonómica,
roles léxicos y semánticos que se sabe que el candidato puede
cumplir, correlaciones entre terminos con la pregunta, popularidad (u
obscuridad) de la fuente del pasaje, aliases, etc.

POR EJEMPLO: COPIAR NIXON

Los distintos scores se combinan luego en un score único para cada
dimensión.

(Merge)

Recién después de este momento, Watson realiza un merge entre
hipótesis idénticas. Las hipótesis idénticas son diferentes
formulaciones ling\"uisticas de lo mismo, por ejemplo:
{\textquotedblleft}X nació en 1928{\textquotedblright} o
{\textquotedblleft}El año de nacimiento de X es
1928{\textquotedblright}. Finalmente, se procede a estimar un ranking
único y una confiabilidad única para las distintas hipótesis. En
este paso se utilizan técnicas de machine learning que requieren
entrenamiento, y modelos basados en scores. Se utilizan técnicas
jerárquicas como mixture of experts y stacked generalization y,
finalmente, un metalearner fue entrenado para ensamblar los distintos
resultados intermedios. 


\bigskip

\subsubsection*{Tiempos y escala}

DeepQA utiliza Apache UIMA, un framework que implementa UIMA
(Unestructured Information Management Architecture): todos los
componentes de DeepQA son IUMA-annotators, módulos que producen
anotaciones y aserciones sobre un texto.

La implementación inicial de Watson corría sobre un sólo
procesador y demoraba aproximadamente 2 horas en contestar una sola
pregunta. La arquitectura paralela permite, sin embargo, que al
correrlo sobre 2500 núcleos -de la implementación final- los
tiempos de respuesta oscilen entre 3 y 5 segundos, que es lo esperado.

Finalmente, la implementación de Watson logró alcanzar el estándar
de resultados de los campeones de Jeopardy y, como ya dijimos,
compitió y ganó el programa en Febrero de 2011. Además, se
realizaron adaptaciones para trabajar sobre los problemas de TREC, en
los cuales se demostró una amplia mejoría en comparación con
PIQUANT y OpenEphyra


\bigskip

\subsubsection*{Conclusiones sobre IBM-Watson}

System level approach.


\bigskip


